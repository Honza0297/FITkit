\documentclass{howto}
\usepackage{ltxmarkup}

\title{Documenting Python}

\makeindex

\author{Guido van Rossum\\
	Fred L. Drake, Jr., editor}
\authoraddress{
	\strong{Python Software Foundation}\\
	Email: \email{docs@python.org}
}

\date{19th September, 2006}			% XXX update before final release!
\input{patchlevel}		% include Python version information


% Now override the stuff that includes author information;
% Guido did *not* write this one!

\author{Fred L. Drake, Jr.}
\authoraddress{
        PythonLabs \\
        Email: \email{fdrake@acm.org}
}


\begin{document}

\maketitle

\begin{abstract}
\noindent
The Python language has a substantial body of
documentation, much of it contributed by various authors.  The markup
used for the Python documentation is based on \LaTeX{} and requires a
significant set of macros written specifically for documenting Python.
This document describes the macros introduced to support Python
documentation and how they should be used to support a wide range of
output formats.

This document describes the document classes and special markup used
in the Python documentation.  Authors may use this guide, in
conjunction with the template files provided with the
distribution, to create or maintain whole documents or sections.

If you're interested in contributing to Python's documentation,
there's no need to learn \LaTeX{} if you're not so inclined; plain
text contributions are more than welcome as well.
\end{abstract}

\tableofcontents


\section{Introduction \label{intro}}

  Python's documentation has long been considered to be good for a
  free programming language.  There are a number of reasons for this,
  the most important being the early commitment of Python's creator,
  Guido van Rossum, to providing documentation on the language and its
  libraries, and the continuing involvement of the user community in
  providing assistance for creating and maintaining documentation.

  The involvement of the community takes many forms, from authoring to
  bug reports to just plain complaining when the documentation could
  be more complete or easier to use.  All of these forms of input from
  the community have proved useful during the time I've been involved
  in maintaining the documentation.

  This document is aimed at authors and potential authors of
  documentation for Python.  More specifically, it is for people
  contributing to the standard documentation and developing additional
  documents using the same tools as the standard documents.  This
  guide will be less useful for authors using the Python documentation
  tools for topics other than Python, and less useful still for
  authors not using the tools at all.

  The material in this guide is intended to assist authors using the
  Python documentation tools.  It includes information on the source
  distribution of the standard documentation, a discussion of the
  document types, reference material on the markup defined in the
  document classes, a list of the external tools needed for processing
  documents, and reference material on the tools provided with the
  documentation resources.  At the end, there is also a section
  discussing future directions for the Python documentation and where
  to turn for more information.

  If your interest is in contributing to the Python documentation, but
  you don't have the time or inclination to learn \LaTeX{} and the
  markup structures documented here, there's a welcoming place for you
  among the Python contributors as well.  Any time you feel that you
  can clarify existing documentation or provide documentation that's
  missing, the existing documentation team will gladly work with you
  to integrate your text, dealing with the markup for you.  Please
  don't let the material in this document stand between the
  documentation and your desire to help out!

\section{Directory Structure \label{directories}}

  The source distribution for the standard Python documentation
  contains a large number of directories.  While third-party documents
  do not need to be placed into this structure or need to be placed
  within a similar structure, it can be helpful to know where to look
  for examples and tools when developing new documents using the
  Python documentation tools.  This section describes this directory
  structure.

  The documentation sources are usually placed within the Python
  source distribution as the top-level directory \file{Doc/}, but
  are not dependent on the Python source distribution in any way.

  The \file{Doc/} directory contains a few files and several
  subdirectories.  The files are mostly self-explanatory, including a
  \file{README} and a \file{Makefile}.  The directories fall into
  three categories:

  \begin{definitions}
    \term{Document Sources}
        The \LaTeX{} sources for each document are placed in a
        separate directory.  These directories are given short
        names which vaguely indicate the document in each:

        \begin{tableii}{p{.75in}|p{3in}}{filenq}{Directory}{Document Title}
          \lineii{api/}
            {\citetitle[../api/api.html]{The Python/C API}}
          \lineii{dist/}
            {\citetitle[../dist/dist.html]{Distributing Python Modules}}
          \lineii{doc/}
            {\citetitle[../doc/doc.html]{Documenting Python}}
          \lineii{ext/}
            {\citetitle[../ext/ext.html]
                       {Extending and Embedding the Python Interpreter}}
          \lineii{inst/}
            {\citetitle[../inst/inst.html]{Installing Python Modules}}
          \lineii{lib/}
            {\citetitle[../lib/lib.html]{Python Library Reference}}
          \lineii{mac/}
            {\citetitle[../mac/mac.html]{Macintosh Module Reference}}
          \lineii{ref/}
            {\citetitle[../ref/ref.html]{Python Reference Manual}}
          \lineii{tut/}
            {\citetitle[../tut/tut.html]{Python Tutorial}}
          \lineii{whatsnew/}
            {\citetitle[../whatsnew/whatsnew24.html]
                       {What's New in Python \shortversion}}
        \end{tableii}

    \term{Format-Specific Output}
        Most output formats have a directory which contains a
        \file{Makefile} which controls the generation of that format
        and provides storage for the formatted documents.  The only
        variations within this category are the Portable Document
        Format (PDF) and PostScript versions are placed in the
        directories \file{paper-a4/} and \file{paper-letter/} (this
        causes all the temporary files created by \LaTeX{} to be kept
        in the same place for each paper size, where they can be more
        easily ignored).

        \begin{tableii}{p{.75in}|p{3in}}{filenq}{Directory}{Output Formats}
          \lineii{html/}{HTML output}
          \lineii{info/}{GNU info output}
          \lineii{isilo/}{\ulink{iSilo}{http://www.isilo.com/}
                          documents (for Palm OS devices)}
          \lineii{paper-a4/}{PDF and PostScript, A4 paper}
          \lineii{paper-letter/}{PDF and PostScript, US-Letter paper}
        \end{tableii}

    \term{Supplemental Files}
        Some additional directories are used to store supplemental
        files used for the various processes.  Directories are
        included for the shared \LaTeX{} document classes, the
        \LaTeX2HTML support, template files for various document
        components, and the scripts used to perform various steps in
        the formatting processes.

        \begin{tableii}{p{.75in}|p{3in}}{filenq}{Directory}{Contents}
          \lineii{commontex/}{Document content shared among documents}
          \lineii{perl/}     {Support for \LaTeX2HTML processing}
          \lineii{templates/}{Example files for source documents}
          \lineii{texinputs/}{Style implementation for \LaTeX}
          \lineii{tools/}    {Custom processing scripts}
        \end{tableii}

  \end{definitions}


\section{Style Guide \label{style-guide}}

  The Python documentation should follow the \citetitle
  [http://developer.apple.com/documentation/UserExperience/Conceptual/APStyleGuide/AppleStyleGuide2003.pdf]
  {Apple Publications Style Guide} wherever possible.  This particular
  style guide was selected mostly because it seems reasonable and is
  easy to get online.

  Topics which are not covered in the Apple's style guide will be
  discussed in this document if necessary.

  Footnotes are generally discouraged due to the pain of using
  footnotes in the HTML conversion of documents.  Footnotes may be
  used when they are the best way to present specific information.
  When a footnote reference is added at the end of the sentence, it
  should follow the sentence-ending punctuation.  The \LaTeX{} markup
  should appear something like this:

\begin{verbatim}
This sentence has a footnote reference.%
  \footnote{This is the footnote text.}
\end{verbatim}

  Footnotes may appear in the middle of sentences where appropriate.

  Many special names are used in the Python documentation, including
  the names of operating systems, programming languages, standards
  bodies, and the like.  Many of these were assigned \LaTeX{} macros
  at some point in the distant past, and these macros lived on long
  past their usefulness.  In the current markup, most of these entities
  are not assigned any special markup, but the preferred spellings are
  given here to aid authors in maintaining the consistency of
  presentation in the Python documentation.

  Other terms and words deserve special mention as well; these conventions
  should be used to ensure consistency throughout the documentation:

  \begin{description}
    \item[CPU]
    For ``central processing unit.''  Many style guides say this
    should be spelled out on the first use (and if you must use it,
    do so!).  For the Python documentation, this abbreviation should
    be avoided since there's no reasonable way to predict which occurrence
    will be the first seen by the reader.  It is better to use the
    word ``processor'' instead.

    \item[\POSIX]
        The name assigned to a particular group of standards.  This is
        always uppercase.  Use the macro \macro{POSIX} to represent this
        name.

    \item[Python]
        The name of our favorite programming language is always
        capitalized.

    \item[Unicode]
        The name of a character set and matching encoding.  This is
        always written capitalized.

    \item[\UNIX]
        The name of the operating system developed at AT\&T Bell Labs
        in the early 1970s.  Use the macro \macro{UNIX} to use this
        name.
  \end{description}


\section{\LaTeX{} Primer \label{latex-primer}}

  This section is a brief introduction to \LaTeX{} concepts and
  syntax, to provide authors enough information to author documents
  productively without having to become ``\TeX{}nicians.''  This does
  not teach everything needed to know about writing \LaTeX{} for
  Python documentation; many of the standard ``environments'' are not
  described here (though you will learn how to mark something as an
  environment).

  Perhaps the most important concept to keep in mind while marking up
  Python documentation is that while \TeX{} is unstructured, \LaTeX{} was
  designed as a layer on top of \TeX{} which specifically supports
  structured markup.  The Python-specific markup is intended to extend
  the structure provided by standard \LaTeX{} document classes to
  support additional information specific to Python.

  \LaTeX{} documents contain two parts: the preamble and the body.
  The preamble is used to specify certain metadata about the document
  itself, such as the title, the list of authors, the date, and the
  \emph{class} the document belongs to.  Additional information used
  to control index generation and the use of bibliographic databases
  can also be placed in the preamble.  For most authors, the preamble
  can be most easily created by copying it from an existing document
  and modifying a few key pieces of information.

  The \dfn{class} of a document is used to place a document within a
  broad category of documents and set some fundamental formatting
  properties.  For Python documentation, two classes are used: the
  \code{manual} class and the \code{howto} class.  These classes also
  define the additional markup used to document Python concepts and
  structures.  Specific information about these classes is provided in
  section \ref{classes}, ``Document Classes,'' below.  The first thing
  in the preamble is the declaration of the document's class.

  After the class declaration, a number of \emph{macros} are used to
  provide further information about the document and setup any
  additional markup that is needed.  No output is generated from the
  preamble; it is an error to include free text in the preamble
  because it would cause output.

  The document body follows the preamble.  This contains all the
  printed components of the document marked up structurally.  Generic
  \LaTeX{} structures include hierarchical sections, numbered and
  bulleted lists, and special structures for the document abstract and
  indexes.

  \subsection{Syntax \label{latex-syntax}}

    There are some things that an author of Python documentation needs
    to know about \LaTeX{} syntax.

    A \dfn{comment} is started by the ``percent'' character
    (\character{\%}) and continues through the end of the line
    \emph{and all leading whitespace on the following line}.  This is
    a little different from any programming language I know of, so an
    example is in order:

\begin{verbatim}
This is text.% comment
    This is more text.  % another comment
Still more text.
\end{verbatim}

    The first non-comment character following the first comment is the
    letter \character{T} on the second line; the leading whitespace on
    that line is consumed as part of the first comment.  This means
    that there is no space between the first and second sentences, so
    the period and letter \character{T} will be directly adjacent in
    the typeset document.

    Note also that though the first non-comment character after the
    second comment is the letter \character{S}, there is whitespace
    preceding the comment, so the two sentences are separated as
    expected.

    A \dfn{group} is an enclosure for a collection of text and
    commands which encloses the formatting context and constrains the
    scope of any changes to that context made by commands within the
    group.  Groups can be nested hierarchically.  The formatting
    context includes the font and the definition of additional macros
    (or overrides of macros defined in outer groups).  Syntactically,
    groups are enclosed in braces:

\begin{verbatim}
{text in a group}
\end{verbatim}

    An alternate syntax for a group using brackets, \code{[...]}, is
    used by macros and environment constructors which take optional
    parameters; brackets do not normally hold syntactic significance.
    A degenerate group, containing only one atomic bit of content,
    does not need to have an explicit group, unless it is required to
    avoid ambiguity.  Since Python tends toward the explicit, groups
    are also made explicit in the documentation markup.

    Groups are used only sparingly in the Python documentation, except
    for their use in marking parameters to macros and environments.

    A \dfn{macro} is usually a simple construct which is identified by
    name and can take some number of parameters.  In normal \LaTeX{}
    usage, one of these can be optional.  The markup is introduced
    using the backslash character (\character{\e}), and the name is
    given by alphabetic characters (no digits, hyphens, or
    underscores).  Required parameters should be marked as a group,
    and optional parameters should be marked using the alternate
    syntax for a group.

    For example, a macro which takes a single parameter
    would appear like this:

\begin{verbatim}
\name{parameter}
\end{verbatim}

    A macro which takes an optional parameter would be typed like this
    when the optional parameter is given:

\begin{verbatim}
\name[optional]
\end{verbatim}

    If both optional and required parameters are to be required, it
    looks like this:

\begin{verbatim}
\name[optional]{required}
\end{verbatim}

    A macro name may be followed by a space or newline; a space
    between the macro name and any parameters will be consumed, but
    this usage is not practiced in the Python documentation.  Such a
    space is still consumed if there are no parameters to the macro,
    in which case inserting an empty group (\code{\{\}}) or explicit
    word space (\samp{\e\ }) immediately after the macro name helps to
    avoid running the expansion of the macro into the following text.
    Macros which take no parameters but which should not be followed
    by a word space do not need special treatment if the following
    character in the document source if not a name character (such as
    punctuation).

    Each line of this example shows an appropriate way to write text
    which includes a macro which takes no parameters:

\begin{verbatim}
This \UNIX{} is followed by a space.
This \UNIX\ is also followed by a space.
\UNIX, followed by a comma, needs no additional markup.
\end{verbatim}

    An \dfn{environment} is a larger construct than a macro, and can
    be used for things with more content than would conveniently fit
    in a macro parameter.  They are primarily used when formatting
    parameters need to be changed before and after a large chunk of
    content, but the content itself needs to be highly flexible.  Code
    samples are presented using an environment, and descriptions of
    functions, methods, and classes are also marked using environments.

    Since the content of an environment is free-form and can consist
    of several paragraphs, they are actually marked using a pair of
    macros: \macro{begin} and \macro{end}.  These macros both take the
    name of the environment as a parameter.  An example is the
    environment used to mark the abstract of a document:

\begin{verbatim}
\begin{abstract}
  This is the text of the abstract.  It concisely explains what
  information is found in the document.

  It can consist of multiple paragraphs.
\end{abstract}
\end{verbatim}

    An environment can also have required and optional parameters of
    its own.  These follow the parameter of the \macro{begin} macro.
    This example shows an environment which takes a single required
    parameter:

\begin{verbatim}
\begin{datadesc}{controlnames}
  A 33-element string array that contains the \ASCII{} mnemonics for
  the thirty-two \ASCII{} control characters from 0 (NUL) to 0x1f
  (US), in order, plus the mnemonic \samp{SP} for the space character.
\end{datadesc}
\end{verbatim}

    There are a number of less-used marks in \LaTeX{} which are used
    to enter characters which are not found in \ASCII{} or which a
    considered special, or \emph{active} in \TeX{} or \LaTeX.  Given
    that these are often used adjacent to other characters, the markup
    required to produce the proper character may need to be followed
    by a space or an empty group, or the markup can be enclosed in a
    group.  Some which are found in Python documentation are:

\begin{tableii}{c|l}{textrm}{Character}{Markup}
  \lineii{\textasciicircum}{\code{\e textasciicircum}}
  \lineii{\textasciitilde}{\code{\e textasciitilde}}
  \lineii{\textgreater}{\code{\e textgreater}}
  \lineii{\textless}{\code{\e textless}}
  \lineii{\c c}{\code{\e c c}}
  \lineii{\"o}{\code{\e"o}}
  \lineii{\o}{\code{\e o}}
\end{tableii}


  \subsection{Hierarchical Structure \label{latex-structure}}

    \LaTeX{} expects documents to be arranged in a conventional,
    hierarchical way, with chapters, sections, sub-sections,
    appendixes, and the like.  These are marked using macros rather
    than environments, probably because the end of a section can be
    safely inferred when a section of equal or higher level starts.

    There are six ``levels'' of sectioning in the document classes
    used for Python documentation, and the deepest two
    levels\footnote{The deepest levels have the highest numbers in the
      table.} are not used.  The levels are:

      \begin{tableiii}{c|l|c}{textrm}{Level}{Macro Name}{Notes}
        \lineiii{1}{\macro{chapter}}{(1)}
        \lineiii{2}{\macro{section}}{}
        \lineiii{3}{\macro{subsection}}{}
        \lineiii{4}{\macro{subsubsection}}{}
        \lineiii{5}{\macro{paragraph}}{(2)}
        \lineiii{6}{\macro{subparagraph}}{}
      \end{tableiii}

    \noindent
    Notes:

    \begin{description}
      \item[(1)]
      Only used for the \code{manual} documents, as described in
      section \ref{classes}, ``Document Classes.''
      \item[(2)]
      Not the same as a paragraph of text; nobody seems to use this.
    \end{description}


  \subsection{Common Environments \label{latex-environments}}

    \LaTeX{} provides a variety of environments even without the
    additional markup provided by the Python-specific document classes
    introduced in the next section.  The following environments are
    provided as part of standard \LaTeX{} and are being used in the
    standard Python documentation; descriptions will be added here as
    time allows.

\begin{verbatim}
abstract
alltt
description
displaymath
document
enumerate
figure
flushleft
itemize
list
math
quotation
quote
sloppypar
verbatim
\end{verbatim}


\section{Document Classes \label{classes}}

  Two \LaTeX{} document classes are defined specifically for use with
  the Python documentation.  The \code{manual} class is for large
  documents which are sectioned into chapters, and the \code{howto}
  class is for smaller documents.

  The \code{manual} documents are larger and are used for most of the
  standard documents.  This document class is based on the standard
  \LaTeX{} \code{report} class and is formatted very much like a long
  technical report.  The \citetitle[../ref/ref.html]{Python Reference
  Manual} is a good example of a \code{manual} document, and the
  \citetitle[../lib/lib.html]{Python Library Reference} is a large
  example.

  The \code{howto} documents are shorter, and don't have the large
  structure of the \code{manual} documents.  This class is based on
  the standard \LaTeX{} \code{article} class and is formatted somewhat
  like the Linux Documentation Project's ``HOWTO'' series as done
  originally using the LinuxDoc software.  The original intent for the
  document class was that it serve a similar role as the LDP's HOWTO
  series, but the applicability of the class turns out to be somewhat
  broader.  This class is used for ``how-to'' documents (this
  document is an example) and for shorter reference manuals for small,
  fairly cohesive module libraries.  Examples of the later use include
\citetitle[http://starship.python.net/crew/fdrake/manuals/krb5py/krb5py.html]{Using
  Kerberos from Python}, which contains reference material for an
  extension package.  These documents are roughly equivalent to a
  single chapter from a larger work.


\section{Special Markup Constructs \label{special-constructs}}

  The Python document classes define a lot of new environments and
  macros.  This section contains the reference material for these
  facilities.  Documentation for ``standard'' \LaTeX{} constructs is
  not included here, though they are used in the Python documentation.

  \subsection{Markup for the Preamble \label{preamble-info}}

    \begin{macrodesc}{release}{\p{ver}}
      Set the version number for the software described in the
      document.
    \end{macrodesc}

    \begin{macrodesc}{setshortversion}{\p{sver}}
      Specify the ``short'' version number of the documented software
      to be \var{sver}.
    \end{macrodesc}

  \subsection{Meta-information Markup \label{meta-info}}

    \begin{macrodesc}{sectionauthor}{\p{author}\p{email}}
      Identifies the author of the current section.  \var{author}
      should be the author's name such that it can be used for
      presentation (though it isn't), and \var{email} should be the
      author's email address.  The domain name portion of
      the address should be lower case.

      No presentation is generated from this markup, but it is used to
      help keep track of contributions.
    \end{macrodesc}

  \subsection{Information Units \label{info-units}}

    XXX Explain terminology, or come up with something more ``lay.''

    There are a number of environments used to describe specific
    features provided by modules.  Each environment requires
    parameters needed to provide basic information about what is being
    described, and the environment content should be the description.
    Most of these environments make entries in the general index (if
    one is being produced for the document); if no index entry is
    desired, non-indexing variants are available for many of these
    environments.  The environments have names of the form
    \code{\var{feature}desc}, and the non-indexing variants are named
    \code{\var{feature}descni}.  The available variants are explicitly
    included in the list below.

    For each of these environments, the first parameter, \var{name},
    provides the name by which the feature is accessed.

    Environments which describe features of objects within a module,
    such as object methods or data attributes, allow an optional
    \var{type name} parameter.  When the feature is an attribute of
    class instances, \var{type name} only needs to be given if the
    class was not the most recently described class in the module; the
    \var{name} value from the most recent \env{classdesc} is implied.
    For features of built-in or extension types, the \var{type name}
    value should always be provided.  Another special case includes
    methods and members of general ``protocols,'' such as the
    formatter and writer protocols described for the
    \module{formatter} module: these may be documented without any
    specific implementation classes, and will always require the
    \var{type name} parameter to be provided.

    \begin{envdesc}{cfuncdesc}{\p{type}\p{name}\p{args}}
      Environment used to described a C function.  The \var{type}
      should be specified as a \keyword{typedef} name, \code{struct
      \var{tag}}, or the name of a primitive type.  If it is a pointer
      type, the trailing asterisk should not be preceded by a space.
      \var{name} should be the name of the function (or function-like
      pre-processor macro), and \var{args} should give the types and
      names of the parameters.  The names need to be given so they may
      be used in the description.
    \end{envdesc}

    \begin{envdesc}{cmemberdesc}{\p{container}\p{type}\p{name}}
      Description for a structure member.  \var{container} should be
      the \keyword{typedef} name, if there is one, otherwise if should
      be \samp{struct \var{tag}}.  The type of the member should given
      as \var{type}, and the name should be given as \var{name}.  The
      text of the description should include the range of values
      allowed, how the value should be interpreted, and whether the
      value can be changed.  References to structure members in text
      should use the \macro{member} macro.
    \end{envdesc}

    \begin{envdesc}{csimplemacrodesc}{\p{name}}
      Documentation for a ``simple'' macro.  Simple macros are macros
      which are used for code expansion, but which do not take
      arguments so cannot be described as functions.  This is not to
      be used for simple constant definitions.  Examples of its use
      in the Python documentation include
      \csimplemacro{PyObject_HEAD} and
      \csimplemacro{Py_BEGIN_ALLOW_THREADS}.
    \end{envdesc}

    \begin{envdesc}{ctypedesc}{\op{tag}\p{name}}
      Environment used to described a C type.  The \var{name}
      parameter should be the \keyword{typedef} name.  If the type is
      defined as a \keyword{struct} without a \keyword{typedef},
      \var{name} should have the form \code{struct \var{tag}}.
      \var{name} will be added to the index unless \var{tag} is
      provided, in which case \var{tag} will be used instead.
      \var{tag} should not be used for a \keyword{typedef} name.
    \end{envdesc}

    \begin{envdesc}{cvardesc}{\p{type}\p{name}}
      Description of a global C variable.  \var{type} should be the
      \keyword{typedef} name, \code{struct \var{tag}}, or the name of
      a primitive type.  If variable has a pointer type, the trailing
      asterisk should \emph{not} be preceded by a space.
    \end{envdesc}

    \begin{envdesc}{datadesc}{\p{name}}
      This environment is used to document global data in a module,
      including both variables and values used as ``defined
      constants.''  Class and object attributes are not documented
      using this environment.
    \end{envdesc}
    \begin{envdesc}{datadescni}{\p{name}}
      Like \env{datadesc}, but without creating any index entries.
    \end{envdesc}

    \begin{envdesc}{excclassdesc}{\p{name}\p{constructor parameters}}
      Describe an exception defined by a class.  \var{constructor
      parameters} should not include the \var{self} parameter or
      the parentheses used in the call syntax.  To describe an
      exception class without describing the parameters to its
      constructor, use the \env{excdesc} environment.
    \end{envdesc}

    \begin{envdesc}{excdesc}{\p{name}}
      Describe an exception.  In the case of class exceptions, the
      constructor parameters are not described; use \env{excclassdesc}
      to describe an exception class and its constructor.
    \end{envdesc}

    \begin{envdesc}{funcdesc}{\p{name}\p{parameters}}
      Describe a module-level function.  \var{parameters} should
      not include the parentheses used in the call syntax.  Object
      methods are not documented using this environment.  Bound object
      methods placed in the module namespace as part of the public
      interface of the module are documented using this, as they are
      equivalent to normal functions for most purposes.

      The description should include information about the parameters
      required and how they are used (especially whether mutable
      objects passed as parameters are modified), side effects, and
      possible exceptions.  A small example may be provided.
    \end{envdesc}
    \begin{envdesc}{funcdescni}{\p{name}\p{parameters}}
      Like \env{funcdesc}, but without creating any index entries.
    \end{envdesc}

    \begin{envdesc}{classdesc}{\p{name}\p{constructor parameters}}
      Describe a class and its constructor.  \var{constructor
      parameters} should not include the \var{self} parameter or
      the parentheses used in the call syntax.
    \end{envdesc}

    \begin{envdesc}{classdesc*}{\p{name}}
      Describe a class without describing the constructor.  This can
      be used to describe classes that are merely containers for
      attributes or which should never be instantiated or subclassed
      by user code.
    \end{envdesc}

    \begin{envdesc}{memberdesc}{\op{type name}\p{name}}
      Describe an object data attribute.  The description should
      include information about the type of the data to be expected
      and whether it may be changed directly.
    \end{envdesc}
    \begin{envdesc}{memberdescni}{\op{type name}\p{name}}
      Like \env{memberdesc}, but without creating any index entries.
    \end{envdesc}

    \begin{envdesc}{methoddesc}{\op{type name}\p{name}\p{parameters}}
      Describe an object method.  \var{parameters} should not include
      the \var{self} parameter or the parentheses used in the call
      syntax.  The description should include similar information to
      that described for \env{funcdesc}.
    \end{envdesc}
    \begin{envdesc}{methoddescni}{\op{type name}\p{name}\p{parameters}}
      Like \env{methoddesc}, but without creating any index entries.
    \end{envdesc}


  \subsection{Showing Code Examples \label{showing-examples}}

    Examples of Python source code or interactive sessions are
    represented as \env{verbatim} environments.  This environment
    is a standard part of \LaTeX{}.  It is important to only use
    spaces for indentation in code examples since \TeX{} drops tabs
    instead of converting them to spaces.

    Representing an interactive session requires including the prompts
    and output along with the Python code.  No special markup is
    required for interactive sessions.  After the last line of input
    or output presented, there should not be an ``unused'' primary
    prompt; this is an example of what \emph{not} to do:

\begin{verbatim}
>>> 1 + 1
2
>>>
\end{verbatim}

    Within the \env{verbatim} environment, characters special to
    \LaTeX{} do not need to be specially marked in any way.  The entire
    example will be presented in a monospaced font; no attempt at
    ``pretty-printing'' is made, as the environment must work for
    non-Python code and non-code displays.  There should be no blank
    lines at the top or bottom of any \env{verbatim} display.

    Longer displays of verbatim text may be included by storing the
    example text in an external file containing only plain text.  The
    file may be included using the standard \macro{verbatiminput}
    macro; this macro takes a single argument naming the file
    containing the text.  For example, to include the Python source
    file \file{example.py}, use:

\begin{verbatim}
\verbatiminput{example.py}
\end{verbatim}

    Use of \macro{verbatiminput} allows easier use of special editing
    modes for the included file.  The file should be placed in the
    same directory as the \LaTeX{} files for the document.

    The Python Documentation Special Interest Group has discussed a
    number of approaches to creating pretty-printed code displays and
    interactive sessions; see the Doc-SIG area on the Python Web site
    for more information on this topic.


  \subsection{Inline Markup \label{inline-markup}}

    The macros described in this section are used to mark just about
    anything interesting in the document text.  They may be used in
    headings (though anything involving hyperlinks should be avoided
    there) as well as in the body text.

    \begin{macrodesc}{bfcode}{\p{text}}
      Like \macro{code}, but also makes the font bold-face.
    \end{macrodesc}

    \begin{macrodesc}{cdata}{\p{name}}
      The name of a C-language variable.
    \end{macrodesc}

    \begin{macrodesc}{cfunction}{\p{name}}
      The name of a C-language function.  \var{name} should include the
      function name and the trailing parentheses.
    \end{macrodesc}

    \begin{macrodesc}{character}{\p{char}}
      A character when discussing the character rather than a one-byte
      string value.  The character will be typeset as with \macro{samp}.
    \end{macrodesc}

    \begin{macrodesc}{citetitle}{\op{url}\p{title}}
      A title for a referenced publication.  If \var{url} is specified,
      the title will be made into a hyperlink when formatted as HTML.
    \end{macrodesc}

    \begin{macrodesc}{class}{\p{name}}
      A class name; a dotted name may be used.
    \end{macrodesc}

    \begin{macrodesc}{code}{\p{text}}
      A short code fragment or literal constant value.  Typically, it
      should not include any spaces since no quotation marks are
      added.
    \end{macrodesc}

    \begin{macrodesc}{constant}{\p{name}}
      The name of a ``defined'' constant.  This may be a C-language
      \code{\#define} or a Python variable that is not intended to be
      changed.
    \end{macrodesc}

    \begin{macrodesc}{csimplemacro}{\p{name}}
      The name of a ``simple'' macro.  Simple macros are macros
      which are used for code expansion, but which do not take
      arguments so cannot be described as functions.  This is not to
      be used for simple constant definitions.  Examples of its use
      in the Python documentation include
      \csimplemacro{PyObject_HEAD} and
      \csimplemacro{Py_BEGIN_ALLOW_THREADS}.
    \end{macrodesc}

    \begin{macrodesc}{ctype}{\p{name}}
      The name of a C \keyword{typedef} or structure.  For structures
      defined without a \keyword{typedef}, use \code{\e ctype\{struct
      struct_tag\}} to make it clear that the \keyword{struct} is
      required.
    \end{macrodesc}

    \begin{macrodesc}{deprecated}{\p{version}\p{what to do}}
      Declare whatever is being described as being deprecated starting
      with release \var{version}.  The text given as \var{what to do}
      should recommend something to use instead.  It should be
      complete sentences.  The entire deprecation notice will be
      presented as a separate paragraph; it should either precede or
      succeed the description of the deprecated feature.
    \end{macrodesc}

    \begin{macrodesc}{dfn}{\p{term}}
      Mark the defining instance of \var{term} in the text.  (No index
      entries are generated.)
    \end{macrodesc}

    \begin{macrodesc}{e}{}
      Produces a backslash.  This is convenient in \macro{code},
      \macro{file}, and similar macros, and the \env{alltt}
      environment, and is only defined there.  To
      create a backslash in ordinary text (such as the contents of the
      \macro{citetitle} macro), use the standard \macro{textbackslash}
      macro.
    \end{macrodesc}

    \begin{macrodesc}{email}{\p{address}}
      An email address.  Note that this is \emph{not} hyperlinked in
      any of the possible output formats.  The domain name portion of
      the address should be lower case.
    \end{macrodesc}

    \begin{macrodesc}{emph}{\p{text}}
      Emphasized text; this will be presented in an italic font.
    \end{macrodesc}

    \begin{macrodesc}{envvar}{\p{name}}
      An environment variable.  Index entries are generated.
    \end{macrodesc}

    \begin{macrodesc}{exception}{\p{name}}
      The name of an exception.  A dotted name may be used.
    \end{macrodesc}

    \begin{macrodesc}{file}{\p{file or dir}}
      The name of a file or directory.  In the PDF and PostScript
      outputs, single quotes and a font change are used to indicate
      the file name, but no quotes are used in the HTML output.
      \warning{The \macro{file} macro cannot be used in the
      content of a section title due to processing limitations.}
    \end{macrodesc}

    \begin{macrodesc}{filenq}{\p{file or dir}}
      Like \macro{file}, but single quotes are never used.  This can
      be used in conjunction with tables if a column will only contain
      file or directory names.
      \warning{The \macro{filenq} macro cannot be used in the
      content of a section title due to processing limitations.}
    \end{macrodesc}

    \begin{macrodesc}{function}{\p{name}}
      The name of a Python function; dotted names may be used.
    \end{macrodesc}

    \begin{macrodesc}{infinity}{}
      The symbol for mathematical infinity: \infinity.  Some Web
      browsers are not able to render the HTML representation of this
      symbol properly, but support is growing.
    \end{macrodesc}

    \begin{macrodesc}{kbd}{\p{key sequence}}
      Mark a sequence of keystrokes.  What form \var{key sequence}
      takes may depend on platform- or application-specific
      conventions.  When there are no relevant conventions, the names
      of modifier keys should be spelled out, to improve accessibility
      for new users and non-native speakers.  For example, an
      \program{xemacs} key sequence may be marked like
      \code{\e kbd\{C-x C-f\}}, but without reference to a specific
      application or platform, the same sequence should be marked as
      \code{\e kbd\{Control-x Control-f\}}.
    \end{macrodesc}

    \begin{macrodesc}{keyword}{\p{name}}
      The name of a keyword in a programming language.
    \end{macrodesc}

    \begin{macrodesc}{mailheader}{\p{name}}
      The name of an \rfc{822}-style mail header.  This markup does
      not imply that the header is being used in an email message, but
      can be used to refer to any header of the same ``style.''  This
      is also used for headers defined by the various MIME
      specifications.  The header name should be entered in the same
      way it would normally be found in practice, with the
      camel-casing conventions being preferred where there is more
      than one common usage.  The colon which follows the name of the
      header should not be included.
      For example: \code{\e mailheader\{Content-Type\}}.
    \end{macrodesc}

    \begin{macrodesc}{makevar}{\p{name}}
      The name of a \program{make} variable.
    \end{macrodesc}

    \begin{macrodesc}{manpage}{\p{name}\p{section}}
      A reference to a \UNIX{} manual page.
    \end{macrodesc}

    \begin{macrodesc}{member}{\p{name}}
      The name of a data attribute of an object.
    \end{macrodesc}

    \begin{macrodesc}{method}{\p{name}}
      The name of a method of an object.  \var{name} should include the
      method name and the trailing parentheses.  A dotted name may be
      used.
    \end{macrodesc}

    \begin{macrodesc}{mimetype}{\p{name}}
      The name of a MIME type, or a component of a MIME type (the
      major or minor portion, taken alone).
    \end{macrodesc}

    \begin{macrodesc}{module}{\p{name}}
       The name of a module; a dotted name may be used.  This should
       also be used for package names.
    \end{macrodesc}

    \begin{macrodesc}{newsgroup}{\p{name}}
      The name of a Usenet newsgroup.
    \end{macrodesc}

    \begin{macrodesc}{note}{\p{text}}
      An especially important bit of information about an API that a
      user should be aware of when using whatever bit of API the
      note pertains to.  This should be the last thing in the
      paragraph as the end of the note is not visually marked in
      any way.  The content of \var{text} should be written in
      complete sentences and include all appropriate punctuation.
    \end{macrodesc}

    \begin{macrodesc}{pep}{\p{number}}
      A reference to a Python Enhancement Proposal.  This generates
      appropriate index entries.  The text \samp{PEP \var{number}} is
      generated; in the HTML output, this text is a hyperlink to an
      online copy of the specified PEP.
    \end{macrodesc}

    \begin{macrodesc}{plusminus}{}
      The symbol for indicating a value that may take a positive or
      negative value of a specified magnitude, typically represented
      by a plus sign placed over a minus sign.  For example:
      \code{\e plusminus 3\%{}}.
    \end{macrodesc}

    \begin{macrodesc}{program}{\p{name}}
      The name of an executable program.  This may differ from the
      file name for the executable for some platforms.  In particular,
      the \file{.exe} (or other) extension should be omitted for
      Windows programs.
    \end{macrodesc}

    \begin{macrodesc}{programopt}{\p{option}}
      A command-line option to an executable program.  Use this only
      for ``short'' options, and include the leading hyphen.
    \end{macrodesc}

    \begin{macrodesc}{longprogramopt}{\p{option}}
      A long command-line option to an executable program.  This
      should only be used for long option names which will be prefixed
      by two hyphens; the hyphens should not be provided as part of
      \var{option}.
    \end{macrodesc}

    \begin{macrodesc}{refmodule}{\op{key}\p{name}}
      Like \macro{module}, but create a hyperlink to the documentation
      for the named module.  Note that the corresponding
      \macro{declaremodule} must be in the same document.  If the
      \macro{declaremodule} defines a module key different from the
      module name, it must also be provided as \var{key} to the
      \macro{refmodule} macro.
    \end{macrodesc}

    \begin{macrodesc}{regexp}{\p{string}}
      Mark a regular expression.
    \end{macrodesc}

    \begin{macrodesc}{rfc}{\p{number}}
      A reference to an Internet Request for Comments.  This generates
      appropriate index entries.  The text \samp{RFC \var{number}} is
      generated; in the HTML output, this text is a hyperlink to an
      online copy of the specified RFC.
    \end{macrodesc}

    \begin{macrodesc}{samp}{\p{text}}
      A short code sample, but possibly longer than would be given
      using \macro{code}.  Since quotation marks are added, spaces are
      acceptable.
    \end{macrodesc}

    \begin{macrodesc}{shortversion}{}
      The ``short'' version number of the documented software, as
      specified using the \macro{setshortversion} macro in the
      preamble.  For Python, the short version number for a release is
      the first three characters of the \code{sys.version} value.  For
      example, versions 2.0b1 and 2.0.1 both have a short version of
      2.0.  This may not apply for all packages; if
      \macro{setshortversion} is not used, this produces an empty
      expansion.  See also the \macro{version} macro.
    \end{macrodesc}

    \begin{macrodesc}{strong}{\p{text}}
      Strongly emphasized text; this will be presented using a bold
      font.
    \end{macrodesc}

    \begin{macrodesc}{ulink}{\p{text}\p{url}}
      A hypertext link with a target specified by a URL, but for which
      the link text should not be the title of the resource.  For
      resources being referenced by name, use the \macro{citetitle}
      macro.  Not all formatted versions support arbitrary hypertext
      links.  Note that many characters are special to \LaTeX{} and
      this macro does not always do the right thing.  In particular,
      the tilde character (\character{\~}) is mis-handled; encoding it
      as a hex-sequence does work, use \samp{\%7e} in place of the
      tilde character.
    \end{macrodesc}

    \begin{macrodesc}{url}{\p{url}}
      A URL (or URN).  The URL will be presented as text.  In the HTML
      and PDF formatted versions, the URL will also be a hyperlink.
      This can be used when referring to external resources without
      specific titles; references to resources which have titles
      should be marked using the \macro{citetitle} macro.  See the
      comments about special characters in the description of the
      \macro{ulink} macro for special considerations.
    \end{macrodesc}

    \begin{macrodesc}{var}{\p{name}}
      The name of a variable or formal parameter in running text.
    \end{macrodesc}

    \begin{macrodesc}{version}{}
      The version number of the described software, as specified using
      \macro{release} in the preamble.  See also the
      \macro{shortversion} macro.
    \end{macrodesc}

    \begin{macrodesc}{warning}{\p{text}}
      An important bit of information about an API that a user should
      be very aware of when using whatever bit of API the warning
      pertains to.  This should be the last thing in the paragraph as
      the end of the warning is not visually marked in any way.  The
      content of \var{text} should be written in complete sentences
      and include all appropriate punctuation.  This differs from
      \macro{note} in that it is recommended over \macro{note} for
      information regarding security.
    \end{macrodesc}

    The following two macros are used to describe information that's
    associated with changes from one release to another.  For features
    which are described by a single paragraph, these are typically
    added as separate source lines at the end of the paragraph.  When
    adding these to features described by multiple paragraphs, they
    are usually collected in a single separate paragraph after the
    description.  When both \macro{versionadded} and
    \macro{versionchanged} are used, \macro{versionadded} should come
    first; the versions should be listed in chronological order.  Both
    of these should come before availability statements.  The location
    should be selected so the explanation makes sense and may vary as
    needed.

    \begin{macrodesc}{versionadded}{\op{explanation}\p{version}}
      The version of Python which added the described feature to the
      library or C API.  \var{explanation} should be a \emph{brief}
      explanation of the change consisting of a capitalized sentence
      fragment; a period will be appended by the formatting process.
      When this applies to an entire module, it should be placed at
      the top of the module section before any prose.
    \end{macrodesc}

    \begin{macrodesc}{versionchanged}{\op{explanation}\p{version}}
      The version of Python in which the named feature was changed in
      some way (new parameters, changed side effects, etc.).
      \var{explanation} should be a \emph{brief} explanation of the
      change consisting of a capitalized sentence fragment; a
      period will be appended by the formatting process.  This should
      not generally be applied to modules.
    \end{macrodesc}


  \subsection{Miscellaneous Text Markup \label{misc-text-markup}}

  In addition to the inline markup, some additional ``block'' markup
  is defined to make it easier to bring attention to various bits of
  text.  The markup described here serves this purpose, and is
  intended to be used when marking one or more paragraphs or other
  block constructs (such as \env{verbatim} environments).

  \begin{envdesc}{notice}{\op{type}}
    Label some paragraphs as being worthy of additional attention from
    the reader.  What sort of attention is warranted can be indicated
    by specifying the \var{type} of the notice.  The only values
    defined for \var{type} are \code{note} and \code{warning}; these
    are equivalent in intent to the inline markup of the same name.
    If \var{type} is omitted, \code{note} is used.  Additional values
    may be defined in the future.
  \end{envdesc}


  \subsection{Module-specific Markup \label{module-markup}}

  The markup described in this section is used to provide information
  about a module being documented.  Each module should be documented
  in its own \macro{section}.  A typical use of this markup
  appears at the top of that section and might look like this:

\begin{verbatim}
\section{\module{spam} ---
         Access to the SPAM facility}

\declaremodule{extension}{spam}
  \platform{Unix}
\modulesynopsis{Access to the SPAM facility of \UNIX.}
\moduleauthor{Jane Doe}{jane.doe@frobnitz.org}
\end{verbatim}

  Python packages\index{packages} --- collections of modules that can
  be described as a unit --- are documented using the same markup as
  modules.  The name for a module in a package should be typed in
  ``fully qualified'' form (it should include the package name).
  For example, a module ``foo'' in package ``bar'' should be marked as
  \code{\e module\{bar.foo\}}, and the beginning of the reference
  section would appear as:

\begin{verbatim}
\section{\module{bar.foo} ---
         Module from the \module{bar} package}

\declaremodule{extension}{bar.foo}
\modulesynopsis{Nifty module from the \module{bar} package.}
\moduleauthor{Jane Doe}{jane.doe@frobnitz.org}
\end{verbatim}

  Note that the name of a package is also marked using
  \macro{module}.

  \begin{macrodesc}{declaremodule}{\op{key}\p{type}\p{name}}
    Requires two parameters: module type (\samp{standard},
    \samp{builtin}, \samp{extension}, or \samp{}), and the module
    name.  An optional parameter should be given as the basis for the
    module's ``key'' used for linking to or referencing the section.
    The ``key'' should only be given if the module's name contains any
    underscores, and should be the name with the underscores stripped.
    Note that the \var{type} parameter must be one of the values
    listed above or an error will be printed.  For modules which are
    contained in packages, the fully-qualified name should be given as
    \var{name} parameter.  This should be the first thing after the
    \macro{section} used to introduce the module.
  \end{macrodesc}

  \begin{macrodesc}{platform}{\p{specifier}}
    Specifies the portability of the module.  \var{specifier} is a
    comma-separated list of keys that specify what platforms the
    module is available on.  The keys are short identifiers;
    examples that are in use include \samp{IRIX}, \samp{Mac},
    \samp{Windows}, and \samp{Unix}.  It is important to use a key
    which has already been used when applicable.  This is used to
    provide annotations in the Module Index and the HTML and GNU info
    output.
  \end{macrodesc}

  \begin{macrodesc}{modulesynopsis}{\p{text}}
    The \var{text} is a short, ``one line'' description of the
    module that can be used as part of the chapter introduction.
    This is must be placed after \macro{declaremodule}.
    The synopsis is used in building the contents of the table
    inserted as the \macro{localmoduletable}.  No text is
    produced at the point of the markup.
  \end{macrodesc}

  \begin{macrodesc}{moduleauthor}{\p{name}\p{email}}
    This macro is used to encode information about who authored a
    module.  This is currently not used to generate output, but can be
    used to help determine the origin of the module.
  \end{macrodesc}


  \subsection{Library-level Markup \label{library-markup}}

    This markup is used when describing a selection of modules.  For
    example, the \citetitle[../mac/mac.html]{Macintosh Library
    Modules} document uses this to help provide an overview of the
    modules in the collection, and many chapters in the
    \citetitle[../lib/lib.html]{Python Library Reference} use it for
    the same purpose.

  \begin{macrodesc}{localmoduletable}{}
    If a \file{.syn} file exists for the current
    chapter (or for the entire document in \code{howto} documents), a
    \env{synopsistable} is created with the contents loaded from the
    \file{.syn} file.
  \end{macrodesc}


  \subsection{Table Markup \label{table-markup}}

    There are three general-purpose table environments defined which
    should be used whenever possible.  These environments are defined
    to provide tables of specific widths and some convenience for
    formatting.  These environments are not meant to be general
    replacements for the standard \LaTeX{} table environments, but can
    be used for an advantage when the documents are processed using
    the tools for Python documentation processing.  In particular, the
    generated HTML looks good!  There is also an advantage for the
    eventual conversion of the documentation to XML (see section
    \ref{futures}, ``Future Directions'').

    Each environment is named \env{table\var{cols}}, where \var{cols}
    is the number of columns in the table specified in lower-case
    Roman numerals.  Within each of these environments, an additional
    macro, \macro{line\var{cols}}, is defined, where \var{cols}
    matches the \var{cols} value of the corresponding table
    environment.  These are supported for \var{cols} values of
    \code{ii}, \code{iii}, and \code{iv}.  These environments are all
    built on top of the \env{tabular} environment.  Variants based on
    the \env{longtable} environment are also provided.

    Note that all tables in the standard Python documentation use
    vertical lines between columns, and this must be specified in the
    markup for each table.  A general border around the outside of the
    table is not used, but would be the responsibility of the
    processor; the document markup should not include an exterior
    border.

    The \env{longtable}-based variants of the table environments are
    formatted with extra space before and after, so should only be
    used on tables which are long enough that splitting over multiple
    pages is reasonable; tables with fewer than twenty rows should
    never by marked using the long flavors of the table environments.
    The header row is repeated across the top of each part of the
    table.

    \begin{envdesc}{tableii}{\p{colspec}\p{col1font}\p{heading1}\p{heading2}}
      Create a two-column table using the \LaTeX{} column specifier
      \var{colspec}.  The column specifier should indicate vertical
      bars between columns as appropriate for the specific table, but
      should not specify vertical bars on the outside of the table
      (that is considered a stylesheet issue).  The \var{col1font}
      parameter is used as a stylistic treatment of the first column
      of the table: the first column is presented as
      \code{\e\var{col1font}\{column1\}}.  To avoid treating the first
      column specially, \var{col1font} may be \samp{textrm}.  The
      column headings are taken from the values \var{heading1} and
      \var{heading2}.
    \end{envdesc}

    \begin{envdesc}{longtableii}{\unspecified}
      Like \env{tableii}, but produces a table which may be broken
      across page boundaries.  The parameters are the same as for
      \env{tableii}.
    \end{envdesc}

    \begin{macrodesc}{lineii}{\p{column1}\p{column2}}
      Create a single table row within a \env{tableii} or
      \env{longtableii} environment.
      The text for the first column will be generated by applying the
      macro named by the \var{col1font} value when the \env{tableii}
      was opened.
    \end{macrodesc}

    \begin{envdesc}{tableiii}{\p{colspec}\p{col1font}\p{heading1}\p{heading2}\p{heading3}}
      Like the \env{tableii} environment, but with a third column.
      The heading for the third column is given by \var{heading3}.
    \end{envdesc}

    \begin{envdesc}{longtableiii}{\unspecified}
      Like \env{tableiii}, but produces a table which may be broken
      across page boundaries.  The parameters are the same as for
      \env{tableiii}.
    \end{envdesc}

    \begin{macrodesc}{lineiii}{\p{column1}\p{column2}\p{column3}}
      Like the \macro{lineii} macro, but with a third column.  The
      text for the third column is given by \var{column3}.
    \end{macrodesc}

    \begin{envdesc}{tableiv}{\p{colspec}\p{col1font}\p{heading1}\p{heading2}\p{heading3}\p{heading4}}
      Like the \env{tableiii} environment, but with a fourth column.
      The heading for the fourth column is given by \var{heading4}.
    \end{envdesc}

    \begin{envdesc}{longtableiv}{\unspecified}
      Like \env{tableiv}, but produces a table which may be broken
      across page boundaries.  The parameters are the same as for
      \env{tableiv}.
    \end{envdesc}

    \begin{macrodesc}{lineiv}{\p{column1}\p{column2}\p{column3}\p{column4}}
      Like the \macro{lineiii} macro, but with a fourth column.  The
      text for the fourth column is given by \var{column4}.
    \end{macrodesc}

    \begin{envdesc}{tablev}{\p{colspec}\p{col1font}\p{heading1}\p{heading2}\p{heading3}\p{heading4}\p{heading5}}
      Like the \env{tableiv} environment, but with a fifth column.
      The heading for the fifth column is given by \var{heading5}.
    \end{envdesc}

    \begin{envdesc}{longtablev}{\unspecified}
      Like \env{tablev}, but produces a table which may be broken
      across page boundaries.  The parameters are the same as for
      \env{tablev}.
    \end{envdesc}

    \begin{macrodesc}{linev}{\p{column1}\p{column2}\p{column3}\p{column4}\p{column5}}
      Like the \macro{lineiv} macro, but with a fifth column.  The
      text for the fifth column is given by \var{column5}.
    \end{macrodesc}


    An additional table-like environment is \env{synopsistable}.  The
    table generated by this environment contains two columns, and each
    row is defined by an alternate definition of
    \macro{modulesynopsis}.  This environment is not normally used by
    authors, but is created by the \macro{localmoduletable} macro.

    Here is a small example of a table given in the documentation for
    the \module{warnings} module; markup inside the table cells is
    minimal so the markup for the table itself is readily discernable.
    Here is the markup for the table:

\begin{verbatim}
\begin{tableii}{l|l}{exception}{Class}{Description}
  \lineii{Warning}
         {This is the base class of all warning category classes.  It
          is a subclass of \exception{Exception}.}
  \lineii{UserWarning}
         {The default category for \function{warn()}.}
  \lineii{DeprecationWarning}
         {Base category for warnings about deprecated features.}
  \lineii{SyntaxWarning}
         {Base category for warnings about dubious syntactic
          features.}
  \lineii{RuntimeWarning}
         {Base category for warnings about dubious runtime features.}
  \lineii{FutureWarning}
         {Base category for warnings about constructs that will change
         semantically in the future.}
\end{tableii}
\end{verbatim}

    Here is the resulting table:

\begin{tableii}{l|l}{exception}{Class}{Description}
  \lineii{Warning}
         {This is the base class of all warning category classes.  It
          is a subclass of \exception{Exception}.}
  \lineii{UserWarning}
         {The default category for \function{warn()}.}
  \lineii{DeprecationWarning}
         {Base category for warnings about deprecated features.}
  \lineii{SyntaxWarning}
         {Base category for warnings about dubious syntactic
          features.}
  \lineii{RuntimeWarning}
         {Base category for warnings about dubious runtime features.}
\end{tableii}

    Note that the class names are implicitly marked using the
    \macro{exception} macro, since that is given as the \var{col1font}
    value for the \env{tableii} environment.  To create a table using
    different markup for the first column, use \code{textrm} for the
    \var{col1font} value and mark each entry individually.

    To add a horizontal line between vertical sections of a table, use
    the standard \macro{hline} macro between the rows which should be
    separated:

\begin{verbatim}
\begin{tableii}{l|l}{constant}{Language}{Audience}
  \lineii{APL}{Masochists.}
  \lineii{BASIC}{First-time programmers on PC hardware.}
  \lineii{C}{\UNIX{} \&\ Linux kernel developers.}
    \hline
  \lineii{Python}{Everyone!}
\end{tableii}
\end{verbatim}

    Note that not all presentation formats are capable of displaying a
    horizontal rule in this position.  This is how the table looks in
    the format you're reading now:

\begin{tableii}{l|l}{constant}{Language}{Audience}
  \lineii{APL}{Masochists.}
  \lineii{C}{\UNIX{} \&\ Linux kernel developers.}
  \lineii{JavaScript}{Web developers.}
    \hline
  \lineii{Python}{Everyone!}
\end{tableii}


  \subsection{Reference List Markup \label{references}}

    Many sections include a list of references to module documentation
    or external documents.  These lists are created using the
    \env{seealso} or \env{seealso*} environments.  These environments
    define some additional macros to support creating reference
    entries in a reasonable manner.

    The \env{seealso} environment is typically placed in a section
    just before any sub-sections.  This is done to ensure that
    reference links related to the section are not hidden in a
    subsection in the hypertext renditions of the documentation.  For
    the HTML output, it is shown as a ``side bar,'' boxed off from the
    main flow of the text.  The \env{seealso*} environment is
    different in that it should be used when a list of references is
    being presented as part of the primary content; it is not
    specially set off from the text.

    \begin{envdesc}{seealso}{}
      This environment creates a ``See also:'' heading and defines the
      markup used to describe individual references.
    \end{envdesc}

    \begin{envdesc}{seealso*}{}
      This environment is used to create a list of references which
      form part of the main content.  It is not given a special
      header and is not set off from the main flow of the text.  It
      provides the same additional markup used to describe individual
      references.
    \end{envdesc}

    For each of the following macros, \var{why} should be one or more
    complete sentences, starting with a capital letter (unless it
    starts with an identifier, which should not be modified), and
    ending with the appropriate punctuation.

    These macros are only defined within the content of the
    \env{seealso} and \env{seealso*} environments.

    \begin{macrodesc}{seelink}{\p{url}\p{linktext}\p{why}}
      References to specific on-line resources should be given using
      the \macro{seelink} macro if they don't have a meaningful title
      but there is some short description of what's at the end of the
      link.  Online documents which have identifiable titles should be
      referenced using the \macro{seetitle} macro, using the optional
      parameter to that macro to provide the URL.
    \end{macrodesc}

    \begin{macrodesc}{seemodule}{\op{key}\p{name}\p{why}}
      Refer to another module.  \var{why} should be a brief
      explanation of why the reference may be interesting.  The module
      name is given in \var{name}, with the link key given in
      \var{key} if necessary.  In the HTML and PDF conversions, the
      module name will be a hyperlink to the referred-to module.
      \note{The module must be documented in the same
      document (the corresponding \macro{declaremodule} is required).}
    \end{macrodesc}

    \begin{macrodesc}{seepep}{\p{number}\p{title}\p{why}}
      Refer to an Python Enhancement Proposal (PEP).  \var{number}
      should be the official number assigned by the PEP Editor,
      \var{title} should be the human-readable title of the PEP as
      found in the official copy of the document, and \var{why} should
      explain what's interesting about the PEP.  This should be used
      to refer the reader to PEPs which specify interfaces or language
      features relevant to the material in the annotated section of the
      documentation.
    \end{macrodesc}

    \begin{macrodesc}{seerfc}{\p{number}\p{title}\p{why}}
      Refer to an IETF Request for Comments (RFC).  Otherwise very
      similar to \macro{seepep}.  This should be used
      to refer the reader to PEPs which specify protocols or data
      formats relevant to the material in the annotated section of the
      documentation.
    \end{macrodesc}

    \begin{macrodesc}{seetext}{\p{text}}
      Add arbitrary text \var{text} to the ``See also:'' list.  This
      can be used to refer to off-line materials or on-line materials
      using the \macro{url} macro.  This should consist of one or more
      complete sentences.
    \end{macrodesc}

    \begin{macrodesc}{seetitle}{\op{url}\p{title}\p{why}}
      Add a reference to an external document named \var{title}.  If
      \var{url} is given, the title is made a hyperlink in the HTML
      version of the documentation, and displayed below the title in
      the typeset versions of the documentation.
    \end{macrodesc}

    \begin{macrodesc}{seeurl}{\p{url}\p{why}}
      References to specific on-line resources should be given using
      the \macro{seeurl} macro if they don't have a meaningful title.
      Online documents which have identifiable titles should be
      referenced using the \macro{seetitle} macro, using the optional
      parameter to that macro to provide the URL.
    \end{macrodesc}


  \subsection{Index-generating Markup \label{indexing}}

    Effective index generation for technical documents can be very
    difficult, especially for someone familiar with the topic but not
    the creation of indexes.  Much of the difficulty arises in the
    area of terminology: including the terms an expert would use for a
    concept is not sufficient.  Coming up with the terms that a novice
    would look up is fairly difficult for an author who, typically, is
    an expert in the area she is writing on.

    The truly difficult aspects of index generation are not areas with
    which the documentation tools can help.  However, ease
    of producing the index once content decisions are made is within
    the scope of the tools.  Markup is provided which the processing
    software is able to use to generate a variety of kinds of index
    entry with minimal effort.  Additionally, many of the environments
    described in section \ref{info-units}, ``Information Units,'' will
    generate appropriate entries into the general and module indexes.

    The following macro can be used to control the generation of index
    data, and should be used in the document preamble:

    \begin{macrodesc}{makemodindex}{}
      This should be used in the document preamble if a ``Module
      Index'' is desired for a document containing reference material
      on many modules.  This causes a data file
      \code{lib\var{jobname}.idx} to be created from the
      \macro{declaremodule} macros.  This file can be processed by the
      \program{makeindex} program to generate a file which can be
      \macro{input} into the document at the desired location of the
      module index.
    \end{macrodesc}

    There are a number of macros that are useful for adding index
    entries for particular concepts, many of which are specific to
    programming languages or even Python.

    \begin{macrodesc}{bifuncindex}{\p{name}}
      Add an index entry referring to a built-in function named
      \var{name}; parentheses should not be included after
      \var{name}.
    \end{macrodesc}

    \begin{macrodesc}{exindex}{\p{exception}}
      Add a reference to an exception named \var{exception}.  The
      exception should be class-based.
    \end{macrodesc}

    \begin{macrodesc}{kwindex}{\p{keyword}}
      Add a reference to a language keyword (not a keyword parameter
      in a function or method call).
    \end{macrodesc}

    \begin{macrodesc}{obindex}{\p{object type}}
      Add an index entry for a built-in object type.
    \end{macrodesc}

    \begin{macrodesc}{opindex}{\p{operator}}
      Add a reference to an operator, such as \samp{+}.
    \end{macrodesc}

    \begin{macrodesc}{refmodindex}{\op{key}\p{module}}
      Add an index entry for module \var{module}; if \var{module}
      contains an underscore, the optional parameter \var{key} should
      be provided as the same string with underscores removed.  An
      index entry ``\var{module} (module)'' will be generated.  This
      is intended for use with non-standard modules implemented in
      Python.
    \end{macrodesc}

    \begin{macrodesc}{refexmodindex}{\op{key}\p{module}}
      As for \macro{refmodindex}, but the index entry will be
      ``\var{module} (extension module).''  This is intended for use
      with non-standard modules not implemented in Python.
    \end{macrodesc}

    \begin{macrodesc}{refbimodindex}{\op{key}\p{module}}
      As for \macro{refmodindex}, but the index entry will be
      ``\var{module} (built-in module).''  This is intended for use
      with standard modules not implemented in Python.
    \end{macrodesc}

    \begin{macrodesc}{refstmodindex}{\op{key}\p{module}}
      As for \macro{refmodindex}, but the index entry will be
      ``\var{module} (standard module).''  This is intended for use
      with standard modules implemented in Python.
    \end{macrodesc}

    \begin{macrodesc}{stindex}{\p{statement}}
      Add an index entry for a statement type, such as \keyword{print}
      or \keyword{try}/\keyword{finally}.

      XXX Need better examples of difference from \macro{kwindex}.
    \end{macrodesc}


    Additional macros are provided which are useful for conveniently
    creating general index entries which should appear at many places
    in the index by rotating a list of words.  These are simple macros
    that simply use \macro{index} to build some number of index
    entries.  Index entries build using these macros contain both
    primary and secondary text.

    \begin{macrodesc}{indexii}{\p{word1}\p{word2}}
      Build two index entries.  This is exactly equivalent to using
      \code{\e index\{\var{word1}!\var{word2}\}} and
      \code{\e index\{\var{word2}!\var{word1}\}}.
    \end{macrodesc}

    \begin{macrodesc}{indexiii}{\p{word1}\p{word2}\p{word3}}
      Build three index entries.  This is exactly equivalent to using
      \code{\e index\{\var{word1}!\var{word2} \var{word3}\}},
      \code{\e index\{\var{word2}!\var{word3}, \var{word1}\}}, and
      \code{\e index\{\var{word3}!\var{word1} \var{word2}\}}.
    \end{macrodesc}

    \begin{macrodesc}{indexiv}{\p{word1}\p{word2}\p{word3}\p{word4}}
      Build four index entries.  This is exactly equivalent to using
      \code{\e index\{\var{word1}!\var{word2} \var{word3} \var{word4}\}},
      \code{\e index\{\var{word2}!\var{word3} \var{word4}, \var{word1}\}},
      \code{\e index\{\var{word3}!\var{word4}, \var{word1} \var{word2}\}},
      and
      \code{\e index\{\var{word4}!\var{word1} \var{word2} \var{word3}\}}.
    \end{macrodesc}

  \subsection{Grammar Production Displays \label{grammar-displays}}

    Special markup is available for displaying the productions of a
    formal grammar.  The markup is simple and does not attempt to
    model all aspects of BNF (or any derived forms), but provides
    enough to allow context-free grammars to be displayed in a way
    that causes uses of a symbol to be rendered as hyperlinks to the
    definition of the symbol.  There is one environment and a pair of
    macros:

    \begin{envdesc}{productionlist}{\op{language}}
      This environment is used to enclose a group of productions.  The
      two macros are only defined within this environment.  If a
      document describes more than one language, the optional parameter
      \var{language} should be used to distinguish productions between
      languages.  The value of the parameter should be a short name
      that can be used as part of a filename; colons or other
      characters that can't be used in filename across platforms
      should be included.
    \end{envdesc}

    \begin{macrodesc}{production}{\p{name}\p{definition}}
      A production rule in the grammar.  The rule defines the symbol
      \var{name} to be \var{definition}.  \var{name} should not
      contain any markup, and the use of hyphens in a document which
      supports more than one grammar is undefined.  \var{definition}
      may contain \macro{token} macros and any additional content
      needed to describe the grammatical model of \var{symbol}.  Only
      one \macro{production} may be used to define a symbol ---
      multiple definitions are not allowed.
    \end{macrodesc}

    \begin{macrodesc}{token}{\p{name}}
      The name of a symbol defined by a \macro{production} macro, used
      in the \var{definition} of a symbol.  Where possible, this will
      be rendered as a hyperlink to the definition of the symbol
      \var{name}.
    \end{macrodesc}

    Note that the entire grammar does not need to be defined in a
    single \env{productionlist} environment; any number of
    groupings may be used to describe the grammar.  Every use of the
    \macro{token} must correspond to a \macro{production}.

    The following is an example taken from the
    \citetitle[../ref/identifiers.html]{Python Reference Manual}:

\begin{verbatim}
\begin{productionlist}
  \production{identifier}
             {(\token{letter}|"_") (\token{letter} | \token{digit} | "_")*}
  \production{letter}
             {\token{lowercase} | \token{uppercase}}
  \production{lowercase}
             {"a"..."z"}
  \production{uppercase}
             {"A"..."Z"}
  \production{digit}
             {"0"..."9"}
\end{productionlist}
\end{verbatim}


\subsection{Graphical Interface Components \label{gui-markup}}

  The components of graphical interfaces will be assigned markup, but
  most of the specifics have not been determined.

  \begin{macrodesc}{guilabel}{\p{label}}
    Labels presented as part of an interactive user interface should
    be marked using \macro{guilabel}.  This includes labels from
    text-based interfaces such as those created using \code{curses} or
    other text-based libraries.  Any label used in the interface
    should be marked with this macro, including button labels, window
    titles, field names, menu and menu selection names, and even
    values in selection lists.
  \end{macrodesc}

  \begin{macrodesc}{menuselection}{\p{menupath}}
    Menu selections should be marked using a combination of
    \macro{menuselection} and \macro{sub}.  This macro is used to mark
    a complete sequence of menu selections, including selecting
    submenus and choosing a specific operation, or any subsequence of
    such a sequence.  The names of individual selections should be
    separated by occurrences of \macro{sub}.

    For example, to mark the selection ``\menuselection{Start \sub
    Programs}'', use this markup:

\begin{verbatim}
\menuselection{Start \sub Programs}
\end{verbatim}

    When including a selection that includes some trailing indicator,
    such as the ellipsis some operating systems use to indicate that
    the command opens a dialog, the indicator should be omitted from
    the selection name.

    Individual selection names within the \macro{menuselection} should
    not be marked using \macro{guilabel} since that's implied by using
    \macro{menuselection}.
  \end{macrodesc}

  \begin{macrodesc}{sub}{}
    Separator for menu selections that include multiple levels.  This
    macro is only defined within the context of the
    \macro{menuselection} macro.
  \end{macrodesc}


\section{Processing Tools \label{tools}}

  \subsection{External Tools \label{tools-external}}

    Many tools are needed to be able to process the Python
    documentation if all supported formats are required.  This
    section lists the tools used and when each is required.  Consult
    the \file{Doc/README} file to see if there are specific version
    requirements for any of these.

    \begin{description}
      \item[\program{dvips}]
        This program is a typical part of \TeX{} installations.  It is
        used to generate PostScript from the ``device independent''
        \file{.dvi} files.  It is needed for the conversion to
        PostScript.

      \item[\program{emacs}]
        Emacs is the kitchen sink of programmers' editors, and a damn
        fine kitchen sink it is.  It also comes with some of the
        processing needed to support the proper menu structures for
        Texinfo documents when an info conversion is desired.  This is
        needed for the info conversion.  Using \program{xemacs}
        instead of FSF \program{emacs} may lead to instability in the
        conversion, but that's because nobody seems to maintain the
        Emacs Texinfo code in a portable manner.

      \item[\program{latex}]
        \LaTeX{} is a large and extensible macro package by Leslie
        Lamport, based on \TeX, a world-class typesetter by Donald
        Knuth.  It is used for the conversion to PostScript, and is
        needed for the HTML conversion as well (\LaTeX2HTML requires
        one of the intermediate files it creates).

      \item[\program{latex2html}]
        Probably the longest Perl script anyone ever attempted to
        maintain.  This converts \LaTeX{} documents to HTML documents,
        and does a pretty reasonable job.  It is required for the
        conversions to HTML and GNU info.

      \item[\program{lynx}]
        This is a text-mode Web browser which includes an
        HTML-to-plain text conversion.  This is used to convert
        \code{howto} documents to text.

      \item[\program{make}]
        Just about any version should work for the standard documents,
        but GNU \program{make} is required for the experimental
        processes in \file{Doc/tools/sgmlconv/}, at least while
        they're experimental.  This is not required for running the
        \program{mkhowto} script.

      \item[\program{makeindex}]
        This is a standard program for converting \LaTeX{} index data
        to a formatted index; it should be included with all \LaTeX{}
        installations.  It is needed for the PDF and PostScript
        conversions.

      \item[\program{makeinfo}]
        GNU \program{makeinfo} is used to convert Texinfo documents to
        GNU info files.  Since Texinfo is used as an intermediate
        format in the info conversion, this program is needed in that
        conversion.

      \item[\program{pdflatex}]
        pdf\TeX{} is a relatively new variant of \TeX, and is used to
        generate the PDF version of the manuals.  It is typically
        installed as part of most of the large \TeX{} distributions.
        \program{pdflatex} is pdf\TeX{} using the \LaTeX{} format.

      \item[\program{perl}]
        Perl is required for \LaTeX2HTML{} and one of the scripts used
        to post-process \LaTeX2HTML output, as well as the
        HTML-to-Texinfo conversion.  This is required for
        the HTML and GNU info conversions.

      \item[\program{python}]
        Python is used for many of the scripts in the
        \file{Doc/tools/} directory; it is required for all
        conversions.  This shouldn't be a problem if you're interested
        in writing documentation for Python!
    \end{description}


  \subsection{Internal Tools \label{tools-internal}}

    This section describes the various scripts that are used to
    implement various stages of document processing or to orchestrate
    entire build sequences.  Most of these tools are only useful
    in the context of building the standard documentation, but some
    are more general.

    \begin{description}
      \item[\program{mkhowto}]
        This is the primary script used to format third-party
        documents.  It contains all the logic needed to ``get it
        right.''  The proper way to use this script is to make a
        symbolic link to it or run it in place; the actual script file
        must be stored as part of the documentation source tree,
        though it may be used to format documents outside the tree.
        Use \program{mkhowto} \longprogramopt{help} for a list of
        command line options.

        \program{mkhowto} can be used for both \code{howto} and
        \code{manual} class documents.  It is usually a good idea to
        always use the latest version of this tool rather than a
        version from an older source release of Python.  It can be
        used to generate DVI, HTML, PDF, PostScript, and plain text
        documents.  The GNU info and iSilo formats will be supported
        by this script in some future version.

        Use the \longprogramopt{help} option on this script's command
        line to get a summary of options for this script.

        XXX  Need more here.
    \end{description}


  \subsection{Working on Cygwin \label{cygwin}}

    Installing the required tools under Cygwin under Cygwin can be a
    little tedious.  Most of the required packages can be installed
    using Cygwin's graphical installer, while netpbm and \LaTeX2HTML
    must be installed from source. 

    Start with a reasonably modern version of Cygwin.  If you haven't
    upgraded for a few years, now would be a good time.

    Using the Cygwin installer, make sure your Cygwin installation
    includes Perl, Python, and the \TeX{} packages.  Perl and Python
    are located under the \menuselection{Interpreters} heading.  The
    \TeX{} packages are located under the \menuselection{Text}
    heading, and are named \code{tetex-*}.  To ensure that all
    required packages are available, install every \code{tetex}
    package, except \code{tetex-x11}.  (There may be a more minimal
    set, but I've not spent time trying to minimize the installation.) 

    The netpbm package is used by \LaTeX2HTML, and \emph{must} be
    installed before \LaTeX2HTML can be successfully installed, even
    though its features will not be used for most Python
    documentation.  References to download locations are located in
    the \ulink{netpbm README}{http://netpbm.sourceforge.net/README}.
    Install from the latest stable source distribution according to
    the instructions.  (Note that binary packages of netpbm are
    sometimes available, but these may not work correctly with
    \LaTeX2HTML.)

    \LaTeX2HTML can be installed from the source archive, but only
    after munging one of the files in the distribution.  Download the
    source archive from the \LaTeX2HTML website
    \url{http://www.latex2html.org/} (or one of the many alternate
    sites) and unpack it to a build directory. In the top level of
    this build directory there will be a file named \file{L2hos.pm}.
    Open \file{L2hos.pm} in an editor, and near the bottom of the file
    replace the text \code{\$\textasciicircum{}O} with the text
    \code{'unix'}.  Proceed using this command to build and install
    the software:

\begin{verbatim}
% ./configure && make install
\end{verbatim}

    You should now be able to build at least the DVI, HTML, PDF, and
    PostScript versions of the formatted documentation.


\section{Including Graphics \label{graphics}}

  The standard documentation included with Python makes no use of
  diagrams or images; this is intentional.  The outside tools used to
  format the documentation have not always been suited to working with
  graphics.  As the tools have evolved and been improved by their
  maintainers, support for graphics has improved.

  The internal tools, starting with the \program{mkhowto} script, do
  not provide any direct support for graphics.  However,
  \program{mkhowto} will not interfere with graphics support in the
  external tools.

  Experience using graphics together with these tools and the
  \code{howto} and \code{manual} document classes is not extensive,
  but has been known to work.  The basic approach is this:

  \begin{enumerate}
    \item Create the image or graphic using your favorite
          application.

    \item Convert the image to a format supported by the conversion to
          your desired output format.  If you want to generate HTML or
          PostScript, you can convert the image or graphic to
          encapsulated PostScript (a \file{.eps} file); \LaTeX2HTML
          can convert that to a \file{.gif} file; it may be possible
          to provide a \file{.gif} file directly.  If you want to
          generate PDF, you need to provide an ``encapsulated'' PDF
          file.  This can be generated from encapsulated PostScript
          using the \program{epstopdf} tool provided with the te\TeX{}
          distribution on Linux and \UNIX.

    \item In your document, add this line to ``import'' the general
          graphics support package \code{graphicx}:

\begin{verbatim}
\usepackage{graphicx}
\end{verbatim}

    \item Where you want to include your graphic or image, include
          markup similar to this:

\begin{verbatim}
\begin{figure}
  \centering
  \includegraphics[width=5in]{myimage}
  \caption{Description of my image}
\end{figure}
\end{verbatim}

          In particular, note for the \macro{includegraphics} macro
          that no file extension is provided.  If you're only
          interested in one target format, you can include the
          extension of the appropriate input file, but to allow
          support for multiple formats, omitting the extension makes
          life easier.

    \item Run \program{mkhowto} normally.
  \end{enumerate}

  If you're working on systems which support some sort of
  \program{make} facility, you can use that to ensure the intermediate
  graphic formats are kept up to date.  This example shows a
  \file{Makefile} used to format a document containing a diagram
  created using the \program{dia} application:

\begin{verbatim}
default: pdf
all:     html pdf ps

html:   mydoc/mydoc.html
pdf:    mydoc.pdf
ps:     mydoc.ps

mydoc/mydoc.html:  mydoc.tex mygraphic.eps
        mkhowto --html $<

mydoc.pdf:  mydoc.tex mygraphic.pdf
        mkhowto --pdf $<

mydoc.ps:   mydoc.tex mygraphic.eps
        mkhowto --postscript $<

.SUFFIXES: .dia .eps .pdf

.dia.eps:
        dia --nosplash --export $@ $<

.eps.pdf:
        epstopdf $<
\end{verbatim} % $ <-- bow to font-lock


\section{Future Directions \label{futures}}

  The history of the Python documentation is full of changes, most of
  which have been fairly small and evolutionary.  There has been a
  great deal of discussion about making large changes in the markup
  languages and tools used to process the documentation.  This section
  deals with the nature of the changes and what appears to be the most
  likely path of future development.

  \subsection{Structured Documentation \label{structured}}

    Most of the small changes to the \LaTeX{} markup have been made
    with an eye to divorcing the markup from the presentation, making
    both a bit more maintainable.  Over the course of 1998, a large
    number of changes were made with exactly this in mind; previously,
    changes had been made but in a less systematic manner and with
    more concern for not needing to update the existing content.  The
    result has been a highly structured and semantically loaded markup
    language implemented in \LaTeX.  With almost no basic \TeX{} or
    \LaTeX{} markup in use, however, the markup syntax is about the
    only evidence of \LaTeX{} in the actual document sources.

    One side effect of this is that while we've been able to use
    standard ``engines'' for manipulating the documents, such as
    \LaTeX{} and \LaTeX2HTML, most of the actual transformations have
    been created specifically for Python.  The \LaTeX{} document
    classes and \LaTeX2HTML support are both complete implementations
    of the specific markup designed for these documents.

    Combining highly customized markup with the somewhat esoteric
    systems used to process the documents leads us to ask some
    questions:  Can we do this more easily?  and, Can we do this
    better?  After a great deal of discussion with the community, we
    have determined that actively pursuing modern structured
    documentation systems is worth some investment of time.

    There appear to be two real contenders in this arena: the Standard
    General Markup Language (SGML), and the Extensible Markup Language
    (XML).  Both of these standards have advantages and disadvantages,
    and many advantages are shared.

    SGML offers advantages which may appeal most to authors,
    especially those using ordinary text editors.  There are also
    additional abilities to define content models.  A number of
    high-quality tools with demonstrated maturity are available, but
    most are not free; for those which are, portability issues remain
    a problem.

    The advantages of XML include the availability of a large number
    of evolving tools.  Unfortunately, many of the associated
    standards are still evolving, and the tools will have to follow
    along.  This means that developing a robust tool set that uses
    more than the basic XML 1.0 recommendation is not possible in the
    short term.  The promised availability of a wide variety of
    high-quality tools which support some of the most important
    related standards is not immediate.  Many tools are likely to be
    free, and the portability issues of those which are, are not
    expected to be significant.

    It turns out that converting to an XML or SGML system holds
    promise for translators as well; how much can be done to ease the
    burden on translators remains to be seen, and may have some impact
    on the schema and specific technologies used.

    XXX Eventual migration to XML.

    The documentation will be moved to XML in the future, and tools
    are being written which will convert the documentation from the
    current format to something close to a finished version, to the
    extent that the desired information is already present in the
    documentation.  Some XSLT stylesheets have been started for
    presenting a preliminary XML version as HTML, but the results are
    fairly rough.

    The timeframe for the conversion is not clear since there doesn't
    seem to be much time available to work on this, but the apparent
    benefits are growing more substantial at a moderately rapid pace.


  \subsection{Discussion Forums \label{discussion}}

    Discussion of the future of the Python documentation and related
    topics takes place in the Documentation Special Interest Group, or
    ``Doc-SIG.''  Information on the group, including mailing list
    archives and subscription information, is available at
    \url{http://www.python.org/sigs/doc-sig/}.  The SIG is open to all
    interested parties.

    Comments and bug reports on the standard documents should be sent
    to \email{docs@python.org}.  This may include comments
    about formatting, content, grammatical and spelling errors, or
    this document.  You can also send comments on this document
    directly to the author at \email{fdrake@acm.org}.

# Doxyfile 1.6.1

# This file describes the settings to be used by the documentation system
# doxygen (www.doxygen.org) for a project
#
# All text after a hash (#) is considered a comment and will be ignored
# The format is:
#       TAG = value [value, ...]
# For lists items can also be appended using:
#       TAG += value [value, ...]
# Values that contain spaces should be placed between quotes (" ")

#---------------------------------------------------------------------------
# Project related configuration options
#---------------------------------------------------------------------------

# This tag specifies the encoding used for all characters in the config file
# that follow. The default is UTF-8 which is also the encoding used for all
# text before the first occurrence of this tag. Doxygen uses libiconv (or the
# iconv built into libc) for the transcoding. See
# http://www.gnu.org/software/libiconv for the list of possible encodings.

DOXYFILE_ENCODING      = UTF-8

# The PROJECT_NAME tag is a single word (or a sequence of words surrounded
# by quotes) that should identify the project.

PROJECT_NAME           = PythonQt

# The PROJECT_NUMBER tag can be used to enter a project or revision number.
# This could be handy for archiving the generated documentation or
# if some version control system is used.

PROJECT_NUMBER         = 

# The OUTPUT_DIRECTORY tag is used to specify the (relative or absolute)
# base path where the generated documentation will be put.
# If a relative path is entered, it will be relative to the location
# where doxygen was started. If left blank the current directory will be used.

OUTPUT_DIRECTORY       = .

# If the CREATE_SUBDIRS tag is set to YES, then doxygen will create
# 4096 sub-directories (in 2 levels) under the output directory of each output
# format and will distribute the generated files over these directories.
# Enabling this option can be useful when feeding doxygen a huge amount of
# source files, where putting all generated files in the same directory would
# otherwise cause performance problems for the file system.

CREATE_SUBDIRS         = NO

# The OUTPUT_LANGUAGE tag is used to specify the language in which all
# documentation generated by doxygen is written. Doxygen will use this
# information to generate all constant output in the proper language.
# The default language is English, other supported languages are:
# Afrikaans, Arabic, Brazilian, Catalan, Chinese, Chinese-Traditional,
# Croatian, Czech, Danish, Dutch, Esperanto, Farsi, Finnish, French, German,
# Greek, Hungarian, Italian, Japanese, Japanese-en (Japanese with English
# messages), Korean, Korean-en, Lithuanian, Norwegian, Macedonian, Persian,
# Polish, Portuguese, Romanian, Russian, Serbian, Serbian-Cyrilic, Slovak,
# Slovene, Spanish, Swedish, Ukrainian, and Vietnamese.

OUTPUT_LANGUAGE        = English

# If the BRIEF_MEMBER_DESC tag is set to YES (the default) Doxygen will
# include brief member descriptions after the members that are listed in
# the file and class documentation (similar to JavaDoc).
# Set to NO to disable this.

BRIEF_MEMBER_DESC      = YES

# If the REPEAT_BRIEF tag is set to YES (the default) Doxygen will prepend
# the brief description of a member or function before the detailed description.
# Note: if both HIDE_UNDOC_MEMBERS and BRIEF_MEMBER_DESC are set to NO, the
# brief descriptions will be completely suppressed.

REPEAT_BRIEF           = YES

# This tag implements a quasi-intelligent brief description abbreviator
# that is used to form the text in various listings. Each string
# in this list, if found as the leading text of the brief description, will be
# stripped from the text and the result after processing the whole list, is
# used as the annotated text. Otherwise, the brief description is used as-is.
# If left blank, the following values are used ("$name" is automatically
# replaced with the name of the entity): "The $name class" "The $name widget"
# "The $name file" "is" "provides" "specifies" "contains"
# "represents" "a" "an" "the"

ABBREVIATE_BRIEF       =

# If the ALWAYS_DETAILED_SEC and REPEAT_BRIEF tags are both set to YES then
# Doxygen will generate a detailed section even if there is only a brief
# description.

ALWAYS_DETAILED_SEC    = NO

# If the INLINE_INHERITED_MEMB tag is set to YES, doxygen will show all
# inherited members of a class in the documentation of that class as if those
# members were ordinary class members. Constructors, destructors and assignment
# operators of the base classes will not be shown.

INLINE_INHERITED_MEMB  = NO

# If the FULL_PATH_NAMES tag is set to YES then Doxygen will prepend the full
# path before files name in the file list and in the header files. If set
# to NO the shortest path that makes the file name unique will be used.

FULL_PATH_NAMES        = NO

# If the FULL_PATH_NAMES tag is set to YES then the STRIP_FROM_PATH tag
# can be used to strip a user-defined part of the path. Stripping is
# only done if one of the specified strings matches the left-hand part of
# the path. The tag can be used to show relative paths in the file list.
# If left blank the directory from which doxygen is run is used as the
# path to strip.

STRIP_FROM_PATH        =

# The STRIP_FROM_INC_PATH tag can be used to strip a user-defined part of
# the path mentioned in the documentation of a class, which tells
# the reader which header file to include in order to use a class.
# If left blank only the name of the header file containing the class
# definition is used. Otherwise one should specify the include paths that
# are normally passed to the compiler using the -I flag.

STRIP_FROM_INC_PATH    =

# If the SHORT_NAMES tag is set to YES, doxygen will generate much shorter
# (but less readable) file names. This can be useful is your file systems
# doesn't support long names like on DOS, Mac, or CD-ROM.

SHORT_NAMES            = NO

# If the JAVADOC_AUTOBRIEF tag is set to YES then Doxygen
# will interpret the first line (until the first dot) of a JavaDoc-style
# comment as the brief description. If set to NO, the JavaDoc
# comments will behave just like regular Qt-style comments
# (thus requiring an explicit @brief command for a brief description.)

JAVADOC_AUTOBRIEF      = NO

# If the QT_AUTOBRIEF tag is set to YES then Doxygen will
# interpret the first line (until the first dot) of a Qt-style
# comment as the brief description. If set to NO, the comments
# will behave just like regular Qt-style comments (thus requiring
# an explicit \brief command for a brief description.)

QT_AUTOBRIEF           = NO

# The MULTILINE_CPP_IS_BRIEF tag can be set to YES to make Doxygen
# treat a multi-line C++ special comment block (i.e. a block of //! or ///
# comments) as a brief description. This used to be the default behaviour.
# The new default is to treat a multi-line C++ comment block as a detailed
# description. Set this tag to YES if you prefer the old behaviour instead.

MULTILINE_CPP_IS_BRIEF = NO

# If the INHERIT_DOCS tag is set to YES (the default) then an undocumented
# member inherits the documentation from any documented member that it
# re-implements.

INHERIT_DOCS           = YES

# If the SEPARATE_MEMBER_PAGES tag is set to YES, then doxygen will produce
# a new page for each member. If set to NO, the documentation of a member will
# be part of the file/class/namespace that contains it.

SEPARATE_MEMBER_PAGES  = NO

# The TAB_SIZE tag can be used to set the number of spaces in a tab.
# Doxygen uses this value to replace tabs by spaces in code fragments.

TAB_SIZE               = 4

# This tag can be used to specify a number of aliases that acts
# as commands in the documentation. An alias has the form "name=value".
# For example adding "sideeffect=\par Side Effects:\n" will allow you to
# put the command \sideeffect (or @sideeffect) in the documentation, which
# will result in a user-defined paragraph with heading "Side Effects:".
# You can put \n's in the value part of an alias to insert newlines.

ALIASES                =

# Set the OPTIMIZE_OUTPUT_FOR_C tag to YES if your project consists of C
# sources only. Doxygen will then generate output that is more tailored for C.
# For instance, some of the names that are used will be different. The list
# of all members will be omitted, etc.

OPTIMIZE_OUTPUT_FOR_C  = NO

# Set the OPTIMIZE_OUTPUT_JAVA tag to YES if your project consists of Java
# sources only. Doxygen will then generate output that is more tailored for
# Java. For instance, namespaces will be presented as packages, qualified
# scopes will look different, etc.

OPTIMIZE_OUTPUT_JAVA   = NO

# Set the OPTIMIZE_FOR_FORTRAN tag to YES if your project consists of Fortran
# sources only. Doxygen will then generate output that is more tailored for
# Fortran.

OPTIMIZE_FOR_FORTRAN   = NO

# Set the OPTIMIZE_OUTPUT_VHDL tag to YES if your project consists of VHDL
# sources. Doxygen will then generate output that is tailored for
# VHDL.

OPTIMIZE_OUTPUT_VHDL   = NO

# Doxygen selects the parser to use depending on the extension of the files it parses.
# With this tag you can assign which parser to use for a given extension.
# Doxygen has a built-in mapping, but you can override or extend it using this tag.
# The format is ext=language, where ext is a file extension, and language is one of
# the parsers supported by doxygen: IDL, Java, Javascript, C#, C, C++, D, PHP,
# Objective-C, Python, Fortran, VHDL, C, C++. For instance to make doxygen treat
# .inc files as Fortran files (default is PHP), and .f files as C (default is Fortran),
# use: inc=Fortran f=C. Note that for custom extensions you also need to set FILE_PATTERNS otherwise the files are not read by doxygen.

EXTENSION_MAPPING      =

# If you use STL classes (i.e. std::string, std::vector, etc.) but do not want
# to include (a tag file for) the STL sources as input, then you should
# set this tag to YES in order to let doxygen match functions declarations and
# definitions whose arguments contain STL classes (e.g. func(std::string); v.s.
# func(std::string) {}). This also make the inheritance and collaboration
# diagrams that involve STL classes more complete and accurate.

BUILTIN_STL_SUPPORT    = NO

# If you use Microsoft's C++/CLI language, you should set this option to YES to
# enable parsing support.

CPP_CLI_SUPPORT        = NO

# Set the SIP_SUPPORT tag to YES if your project consists of sip sources only.
# Doxygen will parse them like normal C++ but will assume all classes use public
# instead of private inheritance when no explicit protection keyword is present.

SIP_SUPPORT            = NO

# For Microsoft's IDL there are propget and propput attributes to indicate getter
# and setter methods for a property. Setting this option to YES (the default)
# will make doxygen to replace the get and set methods by a property in the
# documentation. This will only work if the methods are indeed getting or
# setting a simple type. If this is not the case, or you want to show the
# methods anyway, you should set this option to NO.

IDL_PROPERTY_SUPPORT   = YES

# If member grouping is used in the documentation and the DISTRIBUTE_GROUP_DOC
# tag is set to YES, then doxygen will reuse the documentation of the first
# member in the group (if any) for the other members of the group. By default
# all members of a group must be documented explicitly.

DISTRIBUTE_GROUP_DOC   = NO

# Set the SUBGROUPING tag to YES (the default) to allow class member groups of
# the same type (for instance a group of public functions) to be put as a
# subgroup of that type (e.g. under the Public Functions section). Set it to
# NO to prevent subgrouping. Alternatively, this can be done per class using
# the \nosubgrouping command.

SUBGROUPING            = YES

# When TYPEDEF_HIDES_STRUCT is enabled, a typedef of a struct, union, or enum
# is documented as struct, union, or enum with the name of the typedef. So
# typedef struct TypeS {} TypeT, will appear in the documentation as a struct
# with name TypeT. When disabled the typedef will appear as a member of a file,
# namespace, or class. And the struct will be named TypeS. This can typically
# be useful for C code in case the coding convention dictates that all compound
# types are typedef'ed and only the typedef is referenced, never the tag name.

TYPEDEF_HIDES_STRUCT   = NO

# The SYMBOL_CACHE_SIZE determines the size of the internal cache use to
# determine which symbols to keep in memory and which to flush to disk.
# When the cache is full, less often used symbols will be written to disk.
# For small to medium size projects (<1000 input files) the default value is
# probably good enough. For larger projects a too small cache size can cause
# doxygen to be busy swapping symbols to and from disk most of the time
# causing a significant performance penality.
# If the system has enough physical memory increasing the cache will improve the
# performance by keeping more symbols in memory. Note that the value works on
# a logarithmic scale so increasing the size by one will rougly double the
# memory usage. The cache size is given by this formula:
# 2^(16+SYMBOL_CACHE_SIZE). The valid range is 0..9, the default is 0,
# corresponding to a cache size of 2^16 = 65536 symbols

SYMBOL_CACHE_SIZE      = 0

#---------------------------------------------------------------------------
# Build related configuration options
#---------------------------------------------------------------------------

# If the EXTRACT_ALL tag is set to YES doxygen will assume all entities in
# documentation are documented, even if no documentation was available.
# Private class members and static file members will be hidden unless
# the EXTRACT_PRIVATE and EXTRACT_STATIC tags are set to YES

EXTRACT_ALL            = YES

# If the EXTRACT_PRIVATE tag is set to YES all private members of a class
# will be included in the documentation.

EXTRACT_PRIVATE        = NO

# If the EXTRACT_STATIC tag is set to YES all static members of a file
# will be included in the documentation.

EXTRACT_STATIC         = YES

# If the EXTRACT_LOCAL_CLASSES tag is set to YES classes (and structs)
# defined locally in source files will be included in the documentation.
# If set to NO only classes defined in header files are included.

EXTRACT_LOCAL_CLASSES  = YES

# This flag is only useful for Objective-C code. When set to YES local
# methods, which are defined in the implementation section but not in
# the interface are included in the documentation.
# If set to NO (the default) only methods in the interface are included.

EXTRACT_LOCAL_METHODS  = NO

# If this flag is set to YES, the members of anonymous namespaces will be
# extracted and appear in the documentation as a namespace called
# 'anonymous_namespace{file}', where file will be replaced with the base
# name of the file that contains the anonymous namespace. By default
# anonymous namespace are hidden.

EXTRACT_ANON_NSPACES   = NO

# If the HIDE_UNDOC_MEMBERS tag is set to YES, Doxygen will hide all
# undocumented members of documented classes, files or namespaces.
# If set to NO (the default) these members will be included in the
# various overviews, but no documentation section is generated.
# This option has no effect if EXTRACT_ALL is enabled.

HIDE_UNDOC_MEMBERS     = NO

# If the HIDE_UNDOC_CLASSES tag is set to YES, Doxygen will hide all
# undocumented classes that are normally visible in the class hierarchy.
# If set to NO (the default) these classes will be included in the various
# overviews. This option has no effect if EXTRACT_ALL is enabled.

HIDE_UNDOC_CLASSES     = NO

# If the HIDE_FRIEND_COMPOUNDS tag is set to YES, Doxygen will hide all
# friend (class|struct|union) declarations.
# If set to NO (the default) these declarations will be included in the
# documentation.

HIDE_FRIEND_COMPOUNDS  = NO

# If the HIDE_IN_BODY_DOCS tag is set to YES, Doxygen will hide any
# documentation blocks found inside the body of a function.
# If set to NO (the default) these blocks will be appended to the
# function's detailed documentation block.

HIDE_IN_BODY_DOCS      = NO

# The INTERNAL_DOCS tag determines if documentation
# that is typed after a \internal command is included. If the tag is set
# to NO (the default) then the documentation will be excluded.
# Set it to YES to include the internal documentation.

INTERNAL_DOCS          = NO

# If the CASE_SENSE_NAMES tag is set to NO then Doxygen will only generate
# file names in lower-case letters. If set to YES upper-case letters are also
# allowed. This is useful if you have classes or files whose names only differ
# in case and if your file system supports case sensitive file names. Windows
# and Mac users are advised to set this option to NO.

CASE_SENSE_NAMES       = YES

# If the HIDE_SCOPE_NAMES tag is set to NO (the default) then Doxygen
# will show members with their full class and namespace scopes in the
# documentation. If set to YES the scope will be hidden.

HIDE_SCOPE_NAMES       = NO

# If the SHOW_INCLUDE_FILES tag is set to YES (the default) then Doxygen
# will put a list of the files that are included by a file in the documentation
# of that file.

SHOW_INCLUDE_FILES     = YES

# If the INLINE_INFO tag is set to YES (the default) then a tag [inline]
# is inserted in the documentation for inline members.

INLINE_INFO            = YES

# If the SORT_MEMBER_DOCS tag is set to YES (the default) then doxygen
# will sort the (detailed) documentation of file and class members
# alphabetically by member name. If set to NO the members will appear in
# declaration order.

SORT_MEMBER_DOCS       = YES

# If the SORT_BRIEF_DOCS tag is set to YES then doxygen will sort the
# brief documentation of file, namespace and class members alphabetically
# by member name. If set to NO (the default) the members will appear in
# declaration order.

SORT_BRIEF_DOCS        = NO

# If the SORT_MEMBERS_CTORS_1ST tag is set to YES then doxygen will sort the (brief and detailed) documentation of class members so that constructors and destructors are listed first. If set to NO (the default) the constructors will appear in the respective orders defined by SORT_MEMBER_DOCS and SORT_BRIEF_DOCS. This tag will be ignored for brief docs if SORT_BRIEF_DOCS is set to NO and ignored for detailed docs if SORT_MEMBER_DOCS is set to NO.

SORT_MEMBERS_CTORS_1ST = NO

# If the SORT_GROUP_NAMES tag is set to YES then doxygen will sort the
# hierarchy of group names into alphabetical order. If set to NO (the default)
# the group names will appear in their defined order.

SORT_GROUP_NAMES       = NO

# If the SORT_BY_SCOPE_NAME tag is set to YES, the class list will be
# sorted by fully-qualified names, including namespaces. If set to
# NO (the default), the class list will be sorted only by class name,
# not including the namespace part.
# Note: This option is not very useful if HIDE_SCOPE_NAMES is set to YES.
# Note: This option applies only to the class list, not to the
# alphabetical list.

SORT_BY_SCOPE_NAME     = NO

# The GENERATE_TODOLIST tag can be used to enable (YES) or
# disable (NO) the todo list. This list is created by putting \todo
# commands in the documentation.

GENERATE_TODOLIST      = NO

# The GENERATE_TESTLIST tag can be used to enable (YES) or
# disable (NO) the test list. This list is created by putting \test
# commands in the documentation.

GENERATE_TESTLIST      = NO

# The GENERATE_BUGLIST tag can be used to enable (YES) or
# disable (NO) the bug list. This list is created by putting \bug
# commands in the documentation.

GENERATE_BUGLIST       = YES

# The GENERATE_DEPRECATEDLIST tag can be used to enable (YES) or
# disable (NO) the deprecated list. This list is created by putting
# \deprecated commands in the documentation.

GENERATE_DEPRECATEDLIST= YES

# The ENABLED_SECTIONS tag can be used to enable conditional
# documentation sections, marked by \if sectionname ... \endif.

ENABLED_SECTIONS       = sourceonly

# The MAX_INITIALIZER_LINES tag determines the maximum number of lines
# the initial value of a variable or define consists of for it to appear in
# the documentation. If the initializer consists of more lines than specified
# here it will be hidden. Use a value of 0 to hide initializers completely.
# The appearance of the initializer of individual variables and defines in the
# documentation can be controlled using \showinitializer or \hideinitializer
# command in the documentation regardless of this setting.

MAX_INITIALIZER_LINES  = 30

# Set the SHOW_USED_FILES tag to NO to disable the list of files generated
# at the bottom of the documentation of classes and structs. If set to YES the
# list will mention the files that were used to generate the documentation.

SHOW_USED_FILES        = YES

# If the sources in your project are distributed over multiple directories
# then setting the SHOW_DIRECTORIES tag to YES will show the directory hierarchy
# in the documentation. The default is NO.

SHOW_DIRECTORIES       = NO

# Set the SHOW_FILES tag to NO to disable the generation of the Files page.
# This will remove the Files entry from the Quick Index and from the
# Folder Tree View (if specified). The default is YES.

SHOW_FILES             = YES

# Set the SHOW_NAMESPACES tag to NO to disable the generation of the
# Namespaces page.
# This will remove the Namespaces entry from the Quick Index
# and from the Folder Tree View (if specified). The default is YES.

SHOW_NAMESPACES        = YES

# The FILE_VERSION_FILTER tag can be used to specify a program or script that
# doxygen should invoke to get the current version for each file (typically from
# the version control system). Doxygen will invoke the program by executing (via
# popen()) the command <command> <input-file>, where <command> is the value of
# the FILE_VERSION_FILTER tag, and <input-file> is the name of an input file
# provided by doxygen. Whatever the program writes to standard output
# is used as the file version. See the manual for examples.

FILE_VERSION_FILTER    =

# The LAYOUT_FILE tag can be used to specify a layout file which will be parsed by
# doxygen. The layout file controls the global structure of the generated output files
# in an output format independent way. The create the layout file that represents
# doxygen's defaults, run doxygen with the -l option. You can optionally specify a
# file name after the option, if omitted DoxygenLayout.xml will be used as the name
# of the layout file.

LAYOUT_FILE            =

#---------------------------------------------------------------------------
# configuration options related to warning and progress messages
#---------------------------------------------------------------------------

# The QUIET tag can be used to turn on/off the messages that are generated
# by doxygen. Possible values are YES and NO. If left blank NO is used.

QUIET                  = YES

# The WARNINGS tag can be used to turn on/off the warning messages that are
# generated by doxygen. Possible values are YES and NO. If left blank
# NO is used.

WARNINGS               = YES

# If WARN_IF_UNDOCUMENTED is set to YES, then doxygen will generate warnings
# for undocumented members. If EXTRACT_ALL is set to YES then this flag will
# automatically be disabled.

WARN_IF_UNDOCUMENTED   = YES

# If WARN_IF_DOC_ERROR is set to YES, doxygen will generate warnings for
# potential errors in the documentation, such as not documenting some
# parameters in a documented function, or documenting parameters that
# don't exist or using markup commands wrongly.

WARN_IF_DOC_ERROR      = YES

# This WARN_NO_PARAMDOC option can be abled to get warnings for
# functions that are documented, but have no documentation for their parameters
# or return value. If set to NO (the default) doxygen will only warn about
# wrong or incomplete parameter documentation, but not about the absence of
# documentation.

WARN_NO_PARAMDOC       = NO

# The WARN_FORMAT tag determines the format of the warning messages that
# doxygen can produce. The string should contain the $file, $line, and $text
# tags, which will be replaced by the file and line number from which the
# warning originated and the warning text. Optionally the format may contain
# $version, which will be replaced by the version of the file (if it could
# be obtained via FILE_VERSION_FILTER)

WARN_FORMAT            = "$file:$line: $text"

# The WARN_LOGFILE tag can be used to specify a file to which warning
# and error messages should be written. If left blank the output is written
# to stderr.

WARN_LOGFILE           = doxygen.log

#---------------------------------------------------------------------------
# configuration options related to the input files
#---------------------------------------------------------------------------

# The INPUT tag can be used to specify the files and/or directories that contain
# documented source files. You may enter file names like "myfile.cpp" or
# directories like "/usr/src/myproject". Separate the files or directories
# with spaces.

INPUT                  = ../src

# This tag can be used to specify the character encoding of the source files
# that doxygen parses. Internally doxygen uses the UTF-8 encoding, which is
# also the default input encoding. Doxygen uses libiconv (or the iconv built
# into libc) for the transcoding. See http://www.gnu.org/software/libiconv for
# the list of possible encodings.

INPUT_ENCODING         = UTF-8

# If the value of the INPUT tag contains directories, you can use the
# FILE_PATTERNS tag to specify one or more wildcard pattern (like *.cpp
# and *.h) to filter out the source-files in the directories. If left
# blank the following patterns are tested:
# *.c *.cc *.cxx *.cpp *.c++ *.java *.ii *.ixx *.ipp *.i++ *.inl *.h *.hh *.hxx
# *.hpp *.h++ *.idl *.odl *.cs *.php *.php3 *.inc *.m *.mm *.py *.f90

FILE_PATTERNS          = *.h

# The RECURSIVE tag can be used to turn specify whether or not subdirectories
# should be searched for input files as well. Possible values are YES and NO.
# If left blank NO is used.

RECURSIVE              = NO

# The EXCLUDE tag can be used to specify files and/or directories that should
# excluded from the INPUT source files. This way you can easily exclude a
# subdirectory from a directory tree whose root is specified with the INPUT tag.

EXCLUDE                =

# The EXCLUDE_SYMLINKS tag can be used select whether or not files or
# directories that are symbolic links (a Unix filesystem feature) are excluded
# from the input.

EXCLUDE_SYMLINKS       = NO

# If the value of the INPUT tag contains directories, you can use the
# EXCLUDE_PATTERNS tag to specify one or more wildcard patterns to exclude
# certain files from those directories. Note that the wildcards are matched
# against the file with absolute path, so to exclude all test directories
# for example use the pattern */test/*

EXCLUDE_PATTERNS       =

# The EXCLUDE_SYMBOLS tag can be used to specify one or more symbol names
# (namespaces, classes, functions, etc.) that should be excluded from the
# output. The symbol name can be a fully qualified name, a word, or if the
# wildcard * is used, a substring. Examples: ANamespace, AClass,
# AClass::ANamespace, ANamespace::*Test

EXCLUDE_SYMBOLS        =

# The EXAMPLE_PATH tag can be used to specify one or more files or
# directories that contain example code fragments that are included (see
# the \include command).

EXAMPLE_PATH           =

# If the value of the EXAMPLE_PATH tag contains directories, you can use the
# EXAMPLE_PATTERNS tag to specify one or more wildcard pattern (like *.cpp
# and *.h) to filter out the source-files in the directories. If left
# blank all files are included.

EXAMPLE_PATTERNS       =

# If the EXAMPLE_RECURSIVE tag is set to YES then subdirectories will be
# searched for input files to be used with the \include or \dontinclude
# commands irrespective of the value of the RECURSIVE tag.
# Possible values are YES and NO. If left blank NO is used.

EXAMPLE_RECURSIVE      = NO

# The IMAGE_PATH tag can be used to specify one or more files or
# directories that contain image that are included in the documentation (see
# the \image command).

IMAGE_PATH             = .

# The INPUT_FILTER tag can be used to specify a program that doxygen should
# invoke to filter for each input file. Doxygen will invoke the filter program
# by executing (via popen()) the command <filter> <input-file>, where <filter>
# is the value of the INPUT_FILTER tag, and <input-file> is the name of an
# input file. Doxygen will then use the output that the filter program writes
# to standard output.
# If FILTER_PATTERNS is specified, this tag will be
# ignored.

INPUT_FILTER           =

# The FILTER_PATTERNS tag can be used to specify filters on a per file pattern
# basis.
# Doxygen will compare the file name with each pattern and apply the
# filter if there is a match.
# The filters are a list of the form:
# pattern=filter (like *.cpp=my_cpp_filter). See INPUT_FILTER for further
# info on how filters are used. If FILTER_PATTERNS is empty, INPUT_FILTER
# is applied to all files.

FILTER_PATTERNS        =

# If the FILTER_SOURCE_FILES tag is set to YES, the input filter (if set using
# INPUT_FILTER) will be used to filter the input files when producing source
# files to browse (i.e. when SOURCE_BROWSER is set to YES).

FILTER_SOURCE_FILES    = NO

#---------------------------------------------------------------------------
# configuration options related to source browsing
#---------------------------------------------------------------------------

# If the SOURCE_BROWSER tag is set to YES then a list of source files will
# be generated. Documented entities will be cross-referenced with these sources.
# Note: To get rid of all source code in the generated output, make sure also
# VERBATIM_HEADERS is set to NO.

SOURCE_BROWSER         = YES

# Setting the INLINE_SOURCES tag to YES will include the body
# of functions and classes directly in the documentation.

INLINE_SOURCES         = YES

# Setting the STRIP_CODE_COMMENTS tag to YES (the default) will instruct
# doxygen to hide any special comment blocks from generated source code
# fragments. Normal C and C++ comments will always remain visible.

STRIP_CODE_COMMENTS    = YES

# If the REFERENCED_BY_RELATION tag is set to YES
# then for each documented function all documented
# functions referencing it will be listed.

REFERENCED_BY_RELATION = YES

# If the REFERENCES_RELATION tag is set to YES
# then for each documented function all documented entities
# called/used by that function will be listed.

REFERENCES_RELATION    = YES

# If the REFERENCES_LINK_SOURCE tag is set to YES (the default)
# and SOURCE_BROWSER tag is set to YES, then the hyperlinks from
# functions in REFERENCES_RELATION and REFERENCED_BY_RELATION lists will
# link to the source code.
# Otherwise they will link to the documentation.

REFERENCES_LINK_SOURCE = YES

# If the USE_HTAGS tag is set to YES then the references to source code
# will point to the HTML generated by the htags(1) tool instead of doxygen
# built-in source browser. The htags tool is part of GNU's global source
# tagging system (see http://www.gnu.org/software/global/global.html). You
# will need version 4.8.6 or higher.

USE_HTAGS              = NO

# If the VERBATIM_HEADERS tag is set to YES (the default) then Doxygen
# will generate a verbatim copy of the header file for each class for
# which an include is specified. Set to NO to disable this.

VERBATIM_HEADERS       = YES

#---------------------------------------------------------------------------
# configuration options related to the alphabetical class index
#---------------------------------------------------------------------------

# If the ALPHABETICAL_INDEX tag is set to YES, an alphabetical index
# of all compounds will be generated. Enable this if the project
# contains a lot of classes, structs, unions or interfaces.

ALPHABETICAL_INDEX     = YES

# If the alphabetical index is enabled (see ALPHABETICAL_INDEX) then
# the COLS_IN_ALPHA_INDEX tag can be used to specify the number of columns
# in which this list will be split (can be a number in the range [1..20])

COLS_IN_ALPHA_INDEX    = 5

# In case all classes in a project start with a common prefix, all
# classes will be put under the same header in the alphabetical index.
# The IGNORE_PREFIX tag can be used to specify one or more prefixes that
# should be ignored while generating the index headers.

IGNORE_PREFIX          = PythonQt

#---------------------------------------------------------------------------
# configuration options related to the HTML output
#---------------------------------------------------------------------------

# If the GENERATE_HTML tag is set to YES (the default) Doxygen will
# generate HTML output.

GENERATE_HTML          = YES

# The HTML_OUTPUT tag is used to specify where the HTML docs will be put.
# If a relative path is entered the value of OUTPUT_DIRECTORY will be
# put in front of it. If left blank `html' will be used as the default path.

HTML_OUTPUT            = html

# The HTML_FILE_EXTENSION tag can be used to specify the file extension for
# each generated HTML page (for example: .htm,.php,.asp). If it is left blank
# doxygen will generate files with .html extension.

HTML_FILE_EXTENSION    = .html

# The HTML_HEADER tag can be used to specify a personal HTML header for
# each generated HTML page. If it is left blank doxygen will generate a
# standard header.

HTML_HEADER            = header.html

# The HTML_FOOTER tag can be used to specify a personal HTML footer for
# each generated HTML page. If it is left blank doxygen will generate a
# standard footer.

HTML_FOOTER            =

# The HTML_STYLESHEET tag can be used to specify a user-defined cascading
# style sheet that is used by each HTML page. It can be used to
# fine-tune the look of the HTML output. If the tag is left blank doxygen
# will generate a default style sheet. Note that doxygen will try to copy
# the style sheet file to the HTML output directory, so don't put your own
# stylesheet in the HTML output directory as well, or it will be erased!

HTML_STYLESHEET        = stylesheet.css

# If the HTML_ALIGN_MEMBERS tag is set to YES, the members of classes,
# files or namespaces will be aligned in HTML using tables. If set to
# NO a bullet list will be used.

HTML_ALIGN_MEMBERS     = YES

# If the HTML_DYNAMIC_SECTIONS tag is set to YES then the generated HTML
# documentation will contain sections that can be hidden and shown after the
# page has loaded. For this to work a browser that supports
# JavaScript and DHTML is required (for instance Mozilla 1.0+, Firefox
# Netscape 6.0+, Internet explorer 5.0+, Konqueror, or Safari).

HTML_DYNAMIC_SECTIONS  = NO

# If the GENERATE_DOCSET tag is set to YES, additional index files
# will be generated that can be used as input for Apple's Xcode 3
# integrated development environment, introduced with OSX 10.5 (Leopard).
# To create a documentation set, doxygen will generate a Makefile in the
# HTML output directory. Running make will produce the docset in that
# directory and running "make install" will install the docset in
# ~/Library/Developer/Shared/Documentation/DocSets so that Xcode will find
# it at startup.
# See http://developer.apple.com/tools/creatingdocsetswithdoxygen.html for more information.

GENERATE_DOCSET        = NO

# When GENERATE_DOCSET tag is set to YES, this tag determines the name of the
# feed. A documentation feed provides an umbrella under which multiple
# documentation sets from a single provider (such as a company or product suite)
# can be grouped.

DOCSET_FEEDNAME        = "Doxygen generated docs"

# When GENERATE_DOCSET tag is set to YES, this tag specifies a string that
# should uniquely identify the documentation set bundle. This should be a
# reverse domain-name style string, e.g. com.mycompany.MyDocSet. Doxygen
# will append .docset to the name.

DOCSET_BUNDLE_ID       = org.doxygen.Project

# If the GENERATE_HTMLHELP tag is set to YES, additional index files
# will be generated that can be used as input for tools like the
# Microsoft HTML help workshop to generate a compiled HTML help file (.chm)
# of the generated HTML documentation.

GENERATE_HTMLHELP      = NO

# If the GENERATE_HTMLHELP tag is set to YES, the CHM_FILE tag can
# be used to specify the file name of the resulting .chm file. You
# can add a path in front of the file if the result should not be
# written to the html output directory.

CHM_FILE               =

# If the GENERATE_HTMLHELP tag is set to YES, the HHC_LOCATION tag can
# be used to specify the location (absolute path including file name) of
# the HTML help compiler (hhc.exe). If non-empty doxygen will try to run
# the HTML help compiler on the generated index.hhp.

HHC_LOCATION           =

# If the GENERATE_HTMLHELP tag is set to YES, the GENERATE_CHI flag
# controls if a separate .chi index file is generated (YES) or that
# it should be included in the master .chm file (NO).

GENERATE_CHI           = NO

# If the GENERATE_HTMLHELP tag is set to YES, the CHM_INDEX_ENCODING
# is used to encode HtmlHelp index (hhk), content (hhc) and project file
# content.

CHM_INDEX_ENCODING     =

# If the GENERATE_HTMLHELP tag is set to YES, the BINARY_TOC flag
# controls whether a binary table of contents is generated (YES) or a
# normal table of contents (NO) in the .chm file.

BINARY_TOC             = NO

# The TOC_EXPAND flag can be set to YES to add extra items for group members
# to the contents of the HTML help documentation and to the tree view.

TOC_EXPAND             = NO

# If the GENERATE_QHP tag is set to YES and both QHP_NAMESPACE and QHP_VIRTUAL_FOLDER
# are set, an additional index file will be generated that can be used as input for
# Qt's qhelpgenerator to generate a Qt Compressed Help (.qch) of the generated
# HTML documentation.

GENERATE_QHP           = NO

# If the QHG_LOCATION tag is specified, the QCH_FILE tag can
# be used to specify the file name of the resulting .qch file.
# The path specified is relative to the HTML output folder.

QCH_FILE               =

# The QHP_NAMESPACE tag specifies the namespace to use when generating
# Qt Help Project output. For more information please see
# http://doc.trolltech.com/qthelpproject.html#namespace

QHP_NAMESPACE          =

# The QHP_VIRTUAL_FOLDER tag specifies the namespace to use when generating
# Qt Help Project output. For more information please see
# http://doc.trolltech.com/qthelpproject.html#virtual-folders

QHP_VIRTUAL_FOLDER     = doc

# If QHP_CUST_FILTER_NAME is set, it specifies the name of a custom filter to add.
# For more information please see
# http://doc.trolltech.com/qthelpproject.html#custom-filters

QHP_CUST_FILTER_NAME   =

# The QHP_CUST_FILT_ATTRS tag specifies the list of the attributes of the custom filter to add.For more information please see
# <a href="http://doc.trolltech.com/qthelpproject.html#custom-filters">Qt Help Project / Custom Filters</a>.

QHP_CUST_FILTER_ATTRS  =

# The QHP_SECT_FILTER_ATTRS tag specifies the list of the attributes this project's
# filter section matches.
# <a href="http://doc.trolltech.com/qthelpproject.html#filter-attributes">Qt Help Project / Filter Attributes</a>.

QHP_SECT_FILTER_ATTRS  =

# If the GENERATE_QHP tag is set to YES, the QHG_LOCATION tag can
# be used to specify the location of Qt's qhelpgenerator.
# If non-empty doxygen will try to run qhelpgenerator on the generated
# .qhp file.

QHG_LOCATION           =

# The DISABLE_INDEX tag can be used to turn on/off the condensed index at
# top of each HTML page. The value NO (the default) enables the index and
# the value YES disables it.

DISABLE_INDEX          = NO

# This tag can be used to set the number of enum values (range [1..20])
# that doxygen will group on one line in the generated HTML documentation.

ENUM_VALUES_PER_LINE   = 4

# The GENERATE_TREEVIEW tag is used to specify whether a tree-like index
# structure should be generated to display hierarchical information.
# If the tag value is set to YES, a side panel will be generated
# containing a tree-like index structure (just like the one that
# is generated for HTML Help). For this to work a browser that supports
# JavaScript, DHTML, CSS and frames is required (i.e. any modern browser).
# Windows users are probably better off using the HTML help feature.

GENERATE_TREEVIEW      = NO

# By enabling USE_INLINE_TREES, doxygen will generate the Groups, Directories,
# and Class Hierarchy pages using a tree view instead of an ordered list.

USE_INLINE_TREES       = NO

# If the treeview is enabled (see GENERATE_TREEVIEW) then this tag can be
# used to set the initial width (in pixels) of the frame in which the tree
# is shown.

TREEVIEW_WIDTH         = 250

# Use this tag to change the font size of Latex formulas included
# as images in the HTML documentation. The default is 10. Note that
# when you change the font size after a successful doxygen run you need
# to manually remove any form_*.png images from the HTML output directory
# to force them to be regenerated.

FORMULA_FONTSIZE       = 10

# When the SEARCHENGINE tag is enable doxygen will generate a search box for the HTML output. The underlying search engine uses javascript
# and DHTML and should work on any modern browser. Note that when using HTML help (GENERATE_HTMLHELP) or Qt help (GENERATE_QHP)
# there is already a search function so this one should typically
# be disabled.

SEARCHENGINE           = NO

#---------------------------------------------------------------------------
# configuration options related to the LaTeX output
#---------------------------------------------------------------------------

# If the GENERATE_LATEX tag is set to YES (the default) Doxygen will
# generate Latex output.

GENERATE_LATEX         = NO

# The LATEX_OUTPUT tag is used to specify where the LaTeX docs will be put.
# If a relative path is entered the value of OUTPUT_DIRECTORY will be
# put in front of it. If left blank `latex' will be used as the default path.

LATEX_OUTPUT           = latex

# The LATEX_CMD_NAME tag can be used to specify the LaTeX command name to be
# invoked. If left blank `latex' will be used as the default command name.

LATEX_CMD_NAME         = latex

# The MAKEINDEX_CMD_NAME tag can be used to specify the command name to
# generate index for LaTeX. If left blank `makeindex' will be used as the
# default command name.

MAKEINDEX_CMD_NAME     = makeindex

# If the COMPACT_LATEX tag is set to YES Doxygen generates more compact
# LaTeX documents. This may be useful for small projects and may help to
# save some trees in general.

COMPACT_LATEX          = NO

# The PAPER_TYPE tag can be used to set the paper type that is used
# by the printer. Possible values are: a4, a4wide, letter, legal and
# executive. If left blank a4wide will be used.

PAPER_TYPE             = a4wide

# The EXTRA_PACKAGES tag can be to specify one or more names of LaTeX
# packages that should be included in the LaTeX output.

EXTRA_PACKAGES         =

# The LATEX_HEADER tag can be used to specify a personal LaTeX header for
# the generated latex document. The header should contain everything until
# the first chapter. If it is left blank doxygen will generate a
# standard header. Notice: only use this tag if you know what you are doing!

LATEX_HEADER           =

# If the PDF_HYPERLINKS tag is set to YES, the LaTeX that is generated
# is prepared for conversion to pdf (using ps2pdf). The pdf file will
# contain links (just like the HTML output) instead of page references
# This makes the output suitable for online browsing using a pdf viewer.

PDF_HYPERLINKS         = NO

# If the USE_PDFLATEX tag is set to YES, pdflatex will be used instead of
# plain latex in the generated Makefile. Set this option to YES to get a
# higher quality PDF documentation.

USE_PDFLATEX           = NO

# If the LATEX_BATCHMODE tag is set to YES, doxygen will add the \\batchmode.
# command to the generated LaTeX files. This will instruct LaTeX to keep
# running if errors occur, instead of asking the user for help.
# This option is also used when generating formulas in HTML.

LATEX_BATCHMODE        = NO

# If LATEX_HIDE_INDICES is set to YES then doxygen will not
# include the index chapters (such as File Index, Compound Index, etc.)
# in the output.

LATEX_HIDE_INDICES     = NO

# If LATEX_SOURCE_CODE is set to YES then doxygen will include source code with syntax highlighting in the LaTeX output. Note that which sources are shown also depends on other settings such as SOURCE_BROWSER.

LATEX_SOURCE_CODE      = NO

#---------------------------------------------------------------------------
# configuration options related to the RTF output
#---------------------------------------------------------------------------

# If the GENERATE_RTF tag is set to YES Doxygen will generate RTF output
# The RTF output is optimized for Word 97 and may not look very pretty with
# other RTF readers or editors.

GENERATE_RTF           = NO

# The RTF_OUTPUT tag is used to specify where the RTF docs will be put.
# If a relative path is entered the value of OUTPUT_DIRECTORY will be
# put in front of it. If left blank `rtf' will be used as the default path.

RTF_OUTPUT             = rtf

# If the COMPACT_RTF tag is set to YES Doxygen generates more compact
# RTF documents. This may be useful for small projects and may help to
# save some trees in general.

COMPACT_RTF            = NO

# If the RTF_HYPERLINKS tag is set to YES, the RTF that is generated
# will contain hyperlink fields. The RTF file will
# contain links (just like the HTML output) instead of page references.
# This makes the output suitable for online browsing using WORD or other
# programs which support those fields.
# Note: wordpad (write) and others do not support links.

RTF_HYPERLINKS         = NO

# Load stylesheet definitions from file. Syntax is similar to doxygen's
# config file, i.e. a series of assignments. You only have to provide
# replacements, missing definitions are set to their default value.

RTF_STYLESHEET_FILE    =

# Set optional variables used in the generation of an rtf document.
# Syntax is similar to doxygen's config file.

RTF_EXTENSIONS_FILE    =

#---------------------------------------------------------------------------
# configuration options related to the man page output
#---------------------------------------------------------------------------

# If the GENERATE_MAN tag is set to YES (the default) Doxygen will
# generate man pages

GENERATE_MAN           = NO

# The MAN_OUTPUT tag is used to specify where the man pages will be put.
# If a relative path is entered the value of OUTPUT_DIRECTORY will be
# put in front of it. If left blank `man' will be used as the default path.

MAN_OUTPUT             = man

# The MAN_EXTENSION tag determines the extension that is added to
# the generated man pages (default is the subroutine's section .3)

MAN_EXTENSION          = .3

# If the MAN_LINKS tag is set to YES and Doxygen generates man output,
# then it will generate one additional man file for each entity
# documented in the real man page(s). These additional files
# only source the real man page, but without them the man command
# would be unable to find the correct page. The default is NO.

MAN_LINKS              = NO

#---------------------------------------------------------------------------
# configuration options related to the XML output
#---------------------------------------------------------------------------

# If the GENERATE_XML tag is set to YES Doxygen will
# generate an XML file that captures the structure of
# the code including all documentation.

GENERATE_XML           = NO

# The XML_OUTPUT tag is used to specify where the XML pages will be put.
# If a relative path is entered the value of OUTPUT_DIRECTORY will be
# put in front of it. If left blank `xml' will be used as the default path.

XML_OUTPUT             = xml

# The XML_SCHEMA tag can be used to specify an XML schema,
# which can be used by a validating XML parser to check the
# syntax of the XML files.

XML_SCHEMA             =

# The XML_DTD tag can be used to specify an XML DTD,
# which can be used by a validating XML parser to check the
# syntax of the XML files.

XML_DTD                =

# If the XML_PROGRAMLISTING tag is set to YES Doxygen will
# dump the program listings (including syntax highlighting
# and cross-referencing information) to the XML output. Note that
# enabling this will significantly increase the size of the XML output.

XML_PROGRAMLISTING     = YES

#---------------------------------------------------------------------------
# configuration options for the AutoGen Definitions output
#---------------------------------------------------------------------------

# If the GENERATE_AUTOGEN_DEF tag is set to YES Doxygen will
# generate an AutoGen Definitions (see autogen.sf.net) file
# that captures the structure of the code including all
# documentation. Note that this feature is still experimental
# and incomplete at the moment.

GENERATE_AUTOGEN_DEF   = NO

#---------------------------------------------------------------------------
# configuration options related to the Perl module output
#---------------------------------------------------------------------------

# If the GENERATE_PERLMOD tag is set to YES Doxygen will
# generate a Perl module file that captures the structure of
# the code including all documentation. Note that this
# feature is still experimental and incomplete at the
# moment.

GENERATE_PERLMOD       = NO

# If the PERLMOD_LATEX tag is set to YES Doxygen will generate
# the necessary Makefile rules, Perl scripts and LaTeX code to be able
# to generate PDF and DVI output from the Perl module output.

PERLMOD_LATEX          = NO

# If the PERLMOD_PRETTY tag is set to YES the Perl module output will be
# nicely formatted so it can be parsed by a human reader.
# This is useful
# if you want to understand what is going on.
# On the other hand, if this
# tag is set to NO the size of the Perl module output will be much smaller
# and Perl will parse it just the same.

PERLMOD_PRETTY         = YES

# The names of the make variables in the generated doxyrules.make file
# are prefixed with the string contained in PERLMOD_MAKEVAR_PREFIX.
# This is useful so different doxyrules.make files included by the same
# Makefile don't overwrite each other's variables.

PERLMOD_MAKEVAR_PREFIX =

#---------------------------------------------------------------------------
# Configuration options related to the preprocessor
#---------------------------------------------------------------------------

# If the ENABLE_PREPROCESSING tag is set to YES (the default) Doxygen will
# evaluate all C-preprocessor directives found in the sources and include
# files.

ENABLE_PREPROCESSING   = YES

# If the MACRO_EXPANSION tag is set to YES Doxygen will expand all macro
# names in the source code. If set to NO (the default) only conditional
# compilation will be performed. Macro expansion can be done in a controlled
# way by setting EXPAND_ONLY_PREDEF to YES.

MACRO_EXPANSION        = YES

# If the EXPAND_ONLY_PREDEF and MACRO_EXPANSION tags are both set to YES
# then the macro expansion is limited to the macros specified with the
# PREDEFINED and EXPAND_AS_DEFINED tags.

EXPAND_ONLY_PREDEF     = YES

# If the SEARCH_INCLUDES tag is set to YES (the default) the includes files
# in the INCLUDE_PATH (see below) will be search if a #include is found.

SEARCH_INCLUDES        = YES

# The INCLUDE_PATH tag can be used to specify one or more directories that
# contain include files that are not input files but should be processed by
# the preprocessor.

INCLUDE_PATH           =

# You can use the INCLUDE_FILE_PATTERNS tag to specify one or more wildcard
# patterns (like *.h and *.hpp) to filter out the header-files in the
# directories. If left blank, the patterns specified with FILE_PATTERNS will
# be used.

INCLUDE_FILE_PATTERNS  = *.h

# The PREDEFINED tag can be used to specify one or more macro names that
# are defined before the preprocessor is started (similar to the -D option of
# gcc). The argument of the tag is a list of macros of the form: name
# or name=definition (no spaces). If the definition and the = are
# omitted =1 is assumed. To prevent a macro definition from being
# undefined via #undef or recursively expanded use the := operator
# instead of the = operator.

PREDEFINED             =

# If the MACRO_EXPANSION and EXPAND_ONLY_PREDEF tags are set to YES then
# this tag can be used to specify a list of macro names that should be expanded.
# The macro definition that is found in the sources will be used.
# Use the PREDEFINED tag if you want to use a different macro definition.

EXPAND_AS_DEFINED      =

# If the SKIP_FUNCTION_MACROS tag is set to YES (the default) then
# doxygen's preprocessor will remove all function-like macros that are alone
# on a line, have an all uppercase name, and do not end with a semicolon. Such
# function macros are typically used for boiler-plate code, and will confuse
# the parser if not removed.

SKIP_FUNCTION_MACROS   = YES

#---------------------------------------------------------------------------
# Configuration::additions related to external references
#---------------------------------------------------------------------------

# The TAGFILES option can be used to specify one or more tagfiles.
# Optionally an initial location of the external documentation
# can be added for each tagfile. The format of a tag file without
# this location is as follows:
#
# TAGFILES = file1 file2 ...
# Adding location for the tag files is done as follows:
#
# TAGFILES = file1=loc1 "file2 = loc2" ...
# where "loc1" and "loc2" can be relative or absolute paths or
# URLs. If a location is present for each tag, the installdox tool
# does not have to be run to correct the links.
# Note that each tag file must have a unique name
# (where the name does NOT include the path)
# If a tag file is not located in the directory in which doxygen
# is run, you must also specify the path to the tagfile here.

TAGFILES               =

# When a file name is specified after GENERATE_TAGFILE, doxygen will create
# a tag file that is based on the input files it reads.

GENERATE_TAGFILE       =

# If the ALLEXTERNALS tag is set to YES all external classes will be listed
# in the class index. If set to NO only the inherited external classes
# will be listed.

ALLEXTERNALS           = NO

# If the EXTERNAL_GROUPS tag is set to YES all external groups will be listed
# in the modules index. If set to NO, only the current project's groups will
# be listed.

EXTERNAL_GROUPS        = YES

# The PERL_PATH should be the absolute path and name of the perl script
# interpreter (i.e. the result of `which perl').

PERL_PATH              = /usr/bin/perl

#---------------------------------------------------------------------------
# Configuration options related to the dot tool
#---------------------------------------------------------------------------

# If the CLASS_DIAGRAMS tag is set to YES (the default) Doxygen will
# generate a inheritance diagram (in HTML, RTF and LaTeX) for classes with base
# or super classes. Setting the tag to NO turns the diagrams off. Note that
# this option is superseded by the HAVE_DOT option below. This is only a
# fallback. It is recommended to install and use dot, since it yields more
# powerful graphs.

CLASS_DIAGRAMS         = YES

# You can define message sequence charts within doxygen comments using the \msc
# command. Doxygen will then run the mscgen tool (see
# http://www.mcternan.me.uk/mscgen/) to produce the chart and insert it in the
# documentation. The MSCGEN_PATH tag allows you to specify the directory where
# the mscgen tool resides. If left empty the tool is assumed to be found in the
# default search path.

MSCGEN_PATH            =

# If set to YES, the inheritance and collaboration graphs will hide
# inheritance and usage relations if the target is undocumented
# or is not a class.

HIDE_UNDOC_RELATIONS   = YES

# If you set the HAVE_DOT tag to YES then doxygen will assume the dot tool is
# available from the path. This tool is part of Graphviz, a graph visualization
# toolkit from AT&T and Lucent Bell Labs. The other options in this section
# have no effect if this option is set to NO (the default)

HAVE_DOT               = NO

# By default doxygen will write a font called FreeSans.ttf to the output
# directory and reference it in all dot files that doxygen generates. This
# font does not include all possible unicode characters however, so when you need
# these (or just want a differently looking font) you can specify the font name
# using DOT_FONTNAME. You need need to make sure dot is able to find the font,
# which can be done by putting it in a standard location or by setting the
# DOTFONTPATH environment variable or by setting DOT_FONTPATH to the directory
# containing the font.

DOT_FONTNAME           = FreeSans

# The DOT_FONTSIZE tag can be used to set the size of the font of dot graphs.
# The default size is 10pt.

DOT_FONTSIZE           = 10

# By default doxygen will tell dot to use the output directory to look for the
# FreeSans.ttf font (which doxygen will put there itself). If you specify a
# different font using DOT_FONTNAME you can set the path where dot
# can find it using this tag.

DOT_FONTPATH           =

# If the CLASS_GRAPH and HAVE_DOT tags are set to YES then doxygen
# will generate a graph for each documented class showing the direct and
# indirect inheritance relations. Setting this tag to YES will force the
# the CLASS_DIAGRAMS tag to NO.

CLASS_GRAPH            = YES

# If the COLLABORATION_GRAPH and HAVE_DOT tags are set to YES then doxygen
# will generate a graph for each documented class showing the direct and
# indirect implementation dependencies (inheritance, containment, and
# class references variables) of the class with other documented classes.

COLLABORATION_GRAPH    = YES

# If the GROUP_GRAPHS and HAVE_DOT tags are set to YES then doxygen
# will generate a graph for groups, showing the direct groups dependencies

GROUP_GRAPHS           = YES

# If the UML_LOOK tag is set to YES doxygen will generate inheritance and
# collaboration diagrams in a style similar to the OMG's Unified Modeling
# Language.

UML_LOOK               = NO

# If set to YES, the inheritance and collaboration graphs will show the
# relations between templates and their instances.

TEMPLATE_RELATIONS     = NO

# If the ENABLE_PREPROCESSING, SEARCH_INCLUDES, INCLUDE_GRAPH, and HAVE_DOT
# tags are set to YES then doxygen will generate a graph for each documented
# file showing the direct and indirect include dependencies of the file with
# other documented files.

INCLUDE_GRAPH          = YES

# If the ENABLE_PREPROCESSING, SEARCH_INCLUDES, INCLUDED_BY_GRAPH, and
# HAVE_DOT tags are set to YES then doxygen will generate a graph for each
# documented header file showing the documented files that directly or
# indirectly include this file.

INCLUDED_BY_GRAPH      = YES

# If the CALL_GRAPH and HAVE_DOT options are set to YES then
# doxygen will generate a call dependency graph for every global function
# or class method. Note that enabling this option will significantly increase
# the time of a run. So in most cases it will be better to enable call graphs
# for selected functions only using the \callgraph command.

CALL_GRAPH             = NO

# If the CALLER_GRAPH and HAVE_DOT tags are set to YES then
# doxygen will generate a caller dependency graph for every global function
# or class method. Note that enabling this option will significantly increase
# the time of a run. So in most cases it will be better to enable caller
# graphs for selected functions only using the \callergraph command.

CALLER_GRAPH           = NO

# If the GRAPHICAL_HIERARCHY and HAVE_DOT tags are set to YES then doxygen
# will graphical hierarchy of all classes instead of a textual one.

GRAPHICAL_HIERARCHY    = YES

# If the DIRECTORY_GRAPH, SHOW_DIRECTORIES and HAVE_DOT tags are set to YES
# then doxygen will show the dependencies a directory has on other directories
# in a graphical way. The dependency relations are determined by the #include
# relations between the files in the directories.

DIRECTORY_GRAPH        = YES

# The DOT_IMAGE_FORMAT tag can be used to set the image format of the images
# generated by dot. Possible values are png, jpg, or gif
# If left blank png will be used.

DOT_IMAGE_FORMAT       = png

# The tag DOT_PATH can be used to specify the path where the dot tool can be
# found. If left blank, it is assumed the dot tool can be found in the path.

DOT_PATH               =

# The DOTFILE_DIRS tag can be used to specify one or more directories that
# contain dot files that are included in the documentation (see the
# \dotfile command).

DOTFILE_DIRS           =

# The DOT_GRAPH_MAX_NODES tag can be used to set the maximum number of
# nodes that will be shown in the graph. If the number of nodes in a graph
# becomes larger than this value, doxygen will truncate the graph, which is
# visualized by representing a node as a red box. Note that doxygen if the
# number of direct children of the root node in a graph is already larger than
# DOT_GRAPH_MAX_NODES then the graph will not be shown at all. Also note
# that the size of a graph can be further restricted by MAX_DOT_GRAPH_DEPTH.

DOT_GRAPH_MAX_NODES    = 50

# The MAX_DOT_GRAPH_DEPTH tag can be used to set the maximum depth of the
# graphs generated by dot. A depth value of 3 means that only nodes reachable
# from the root by following a path via at most 3 edges will be shown. Nodes
# that lay further from the root node will be omitted. Note that setting this
# option to 1 or 2 may greatly reduce the computation time needed for large
# code bases. Also note that the size of a graph can be further restricted by
# DOT_GRAPH_MAX_NODES. Using a depth of 0 means no depth restriction.

MAX_DOT_GRAPH_DEPTH    = 0

# Set the DOT_TRANSPARENT tag to YES to generate images with a transparent
# background. This is disabled by default, because dot on Windows does not
# seem to support this out of the box. Warning: Depending on the platform used,
# enabling this option may lead to badly anti-aliased labels on the edges of
# a graph (i.e. they become hard to read).

DOT_TRANSPARENT        = NO

# Set the DOT_MULTI_TARGETS tag to YES allow dot to generate multiple output
# files in one run (i.e. multiple -o and -T options on the command line). This
# makes dot run faster, but since only newer versions of dot (>1.8.10)
# support this, this feature is disabled by default.

DOT_MULTI_TARGETS      = NO

# If the GENERATE_LEGEND tag is set to YES (the default) Doxygen will
# generate a legend page explaining the meaning of the various boxes and
# arrows in the dot generated graphs.

GENERATE_LEGEND        = YES

# If the DOT_CLEANUP tag is set to YES (the default) Doxygen will
# remove the intermediate dot files that are used to generate
# the various graphs.

DOT_CLEANUP            = YES


\end{document}
