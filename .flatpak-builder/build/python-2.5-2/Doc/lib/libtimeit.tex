\section{\module{timeit} ---
         Measure execution time of small code snippets}

\declaremodule{standard}{timeit}
\modulesynopsis{Measure the execution time of small code snippets.}

\versionadded{2.3}
\index{Benchmarking}
\index{Performance}

This module provides a simple way to time small bits of Python code.
It has both command line as well as callable interfaces.  It avoids a
number of common traps for measuring execution times.  See also Tim
Peters' introduction to the ``Algorithms'' chapter in the
\citetitle{Python Cookbook}, published by O'Reilly.

The module defines the following public class:

\begin{classdesc}{Timer}{\optional{stmt=\code{'pass'}
                         \optional{, setup=\code{'pass'}
                         \optional{, timer=<timer function>}}}}
Class for timing execution speed of small code snippets.

The constructor takes a statement to be timed, an additional statement
used for setup, and a timer function.  Both statements default to
\code{'pass'}; the timer function is platform-dependent (see the
module doc string).  The statements may contain newlines, as long as
they don't contain multi-line string literals.

To measure the execution time of the first statement, use the
\method{timeit()} method.  The \method{repeat()} method is a
convenience to call \method{timeit()} multiple times and return a list
of results.
\end{classdesc}

\begin{methoddesc}{print_exc}{\optional{file=\constant{None}}}
Helper to print a traceback from the timed code.

Typical use:

\begin{verbatim}
    t = Timer(...)       # outside the try/except
    try:
        t.timeit(...)    # or t.repeat(...)
    except:
        t.print_exc()
\end{verbatim}

The advantage over the standard traceback is that source lines in the
compiled template will be displayed.
The optional \var{file} argument directs where the traceback is sent;
it defaults to \code{sys.stderr}.
\end{methoddesc}

\begin{methoddesc}{repeat}{\optional{repeat\code{=3} \optional{,
                           number\code{=1000000}}}}
Call \method{timeit()} a few times.

This is a convenience function that calls the \method{timeit()}
repeatedly, returning a list of results.  The first argument specifies
how many times to call \method{timeit()}.  The second argument
specifies the \var{number} argument for \function{timeit()}.

\begin{notice}
It's tempting to calculate mean and standard deviation from the result
vector and report these.  However, this is not very useful.  In a typical
case, the lowest value gives a lower bound for how fast your machine can run
the given code snippet; higher values in the result vector are typically not
caused by variability in Python's speed, but by other processes interfering
with your timing accuracy.  So the \function{min()} of the result is
probably the only number you should be interested in.  After that, you
should look at the entire vector and apply common sense rather than
statistics.
\end{notice}
\end{methoddesc}

\begin{methoddesc}{timeit}{\optional{number\code{=1000000}}}
Time \var{number} executions of the main statement.
This executes the setup statement once, and then
returns the time it takes to execute the main statement a number of
times, measured in seconds as a float.  The argument is the number of
times through the loop, defaulting to one million.  The main
statement, the setup statement and the timer function to be used are
passed to the constructor.

\begin{notice}
By default, \method{timeit()} temporarily turns off garbage collection
during the timing.  The advantage of this approach is that it makes
independent timings more comparable.  This disadvantage is that GC
may be an important component of the performance of the function being
measured.  If so, GC can be re-enabled as the first statement in the
\var{setup} string.  For example:
\begin{verbatim}
    timeit.Timer('for i in xrange(10): oct(i)', 'gc.enable()').timeit()
\end{verbatim}
\end{notice}
\end{methoddesc}


\subsection{Command Line Interface}

When called as a program from the command line, the following form is used:

\begin{verbatim}
python timeit.py [-n N] [-r N] [-s S] [-t] [-c] [-h] [statement ...]
\end{verbatim}

where the following options are understood:

\begin{description}
\item[-n N/\longprogramopt{number=N}] how many times to execute 'statement'
\item[-r N/\longprogramopt{repeat=N}] how many times to repeat the timer (default 3)
\item[-s S/\longprogramopt{setup=S}] statement to be executed once initially (default
\code{'pass'})
\item[-t/\longprogramopt{time}] use \function{time.time()}
(default on all platforms but Windows)
\item[-c/\longprogramopt{clock}] use \function{time.clock()} (default on Windows)
\item[-v/\longprogramopt{verbose}] print raw timing results; repeat for more digits
precision
\item[-h/\longprogramopt{help}] print a short usage message and exit
\end{description}

A multi-line statement may be given by specifying each line as a
separate statement argument; indented lines are possible by enclosing
an argument in quotes and using leading spaces.  Multiple
\programopt{-s} options are treated similarly.

If \programopt{-n} is not given, a suitable number of loops is
calculated by trying successive powers of 10 until the total time is
at least 0.2 seconds.

The default timer function is platform dependent.  On Windows,
\function{time.clock()} has microsecond granularity but
\function{time.time()}'s granularity is 1/60th of a second; on \UNIX,
\function{time.clock()} has 1/100th of a second granularity and
\function{time.time()} is much more precise.  On either platform, the
default timer functions measure wall clock time, not the CPU time.
This means that other processes running on the same computer may
interfere with the timing.  The best thing to do when accurate timing
is necessary is to repeat the timing a few times and use the best
time.  The \programopt{-r} option is good for this; the default of 3
repetitions is probably enough in most cases.  On \UNIX, you can use
\function{time.clock()} to measure CPU time.

\begin{notice}
  There is a certain baseline overhead associated with executing a
  pass statement.  The code here doesn't try to hide it, but you
  should be aware of it.  The baseline overhead can be measured by
  invoking the program without arguments.
\end{notice}

The baseline overhead differs between Python versions!  Also, to
fairly compare older Python versions to Python 2.3, you may want to
use Python's \programopt{-O} option for the older versions to avoid
timing \code{SET_LINENO} instructions.

\subsection{Examples}

Here are two example sessions (one using the command line, one using
the module interface) that compare the cost of using
\function{hasattr()} vs. \keyword{try}/\keyword{except} to test for
missing and present object attributes.

\begin{verbatim}
% timeit.py 'try:' '  str.__nonzero__' 'except AttributeError:' '  pass'
100000 loops, best of 3: 15.7 usec per loop
% timeit.py 'if hasattr(str, "__nonzero__"): pass'
100000 loops, best of 3: 4.26 usec per loop
% timeit.py 'try:' '  int.__nonzero__' 'except AttributeError:' '  pass'
1000000 loops, best of 3: 1.43 usec per loop
% timeit.py 'if hasattr(int, "__nonzero__"): pass'
100000 loops, best of 3: 2.23 usec per loop
\end{verbatim}

\begin{verbatim}
>>> import timeit
>>> s = """\
... try:
...     str.__nonzero__
... except AttributeError:
...     pass
... """
>>> t = timeit.Timer(stmt=s)
>>> print "%.2f usec/pass" % (1000000 * t.timeit(number=100000)/100000)
17.09 usec/pass
>>> s = """\
... if hasattr(str, '__nonzero__'): pass
... """
>>> t = timeit.Timer(stmt=s)
>>> print "%.2f usec/pass" % (1000000 * t.timeit(number=100000)/100000)
4.85 usec/pass
>>> s = """\
... try:
...     int.__nonzero__
... except AttributeError:
...     pass
... """
>>> t = timeit.Timer(stmt=s)
>>> print "%.2f usec/pass" % (1000000 * t.timeit(number=100000)/100000)
1.97 usec/pass
>>> s = """\
... if hasattr(int, '__nonzero__'): pass
... """
>>> t = timeit.Timer(stmt=s)
>>> print "%.2f usec/pass" % (1000000 * t.timeit(number=100000)/100000)
3.15 usec/pass
\end{verbatim}

To give the \module{timeit} module access to functions you
define, you can pass a \code{setup} parameter which contains an import
statement:

\begin{verbatim}
def test():
    "Stupid test function"
    L = []
    for i in range(100):
        L.append(i)

if __name__=='__main__':
    from timeit import Timer
    t = Timer("test()", "from __main__ import test")
    print t.timeit()
\end{verbatim}
